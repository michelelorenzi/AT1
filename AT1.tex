\documentclass[a4paper, 10pt, oneside, DIV=9, chapterprefix=true, numbers=enddot,bibliography=totoc]{scrbook}

\RedeclareSectionCommand[tocdynnumwidth]{chapter}
\RedeclareSectionCommands[tocdynindent]{section,subsection}
\usepackage{styleAT1}
\usepackage{shortcutsAT1}
\usepackage[normalem]{ulem}
\usepackage[outline]{contour}
\contourlength{2.25pt}

\newcommand{\embrace}[1]{\textup{(}#1\textup{)}}
\newlength{\LETTERheight}
\AtBeginDocument{\settoheight{\LETTERheight}{I}}
\newcommand*{\longrightsquigarrow}[1]{\ \raisebox{0.24\LETTERheight}{\tikz \draw [-to,
		line join=round, line cap=round,
		decorate, decoration={
			zigzag,
			segment length=4,
			amplitude=.9,
			post=lineto,
			post length=0.42ex
		}] (0,0) -- (#1,0);}\ }
	
\newlength{\HeightOfTextstyleOne}
\settoheight{\HeightOfTextstyleOne}{$\mathbf{1}$}
\newlength{\HeightOfScriptstyleOne}
\settoheight{\HeightOfScriptstyleOne}{$\scriptstyle\mathbf{1}$}
\newlength{\HeightOfScriptscriptstyleOne}
\settoheight{\HeightOfScriptscriptstyleOne}{$\scriptscriptstyle\mathbf{1}$}
\newcommand{\FancyOne}[1]{{\tikz[line cap=round,line join=round,line width=0.35*#1/\HeightOfTextstyleOne,scale=#1/\HeightOfTextstyleOne]{\draw (-0.0225,0.205) to (-0.0225,0.02) to[out=270,in=0] (-0.0425,0) to (-0.071,0) to (0.071,0) to (0.0425,0) to[out=180,in=270] (0.0225,0.02) to (0.0225,0.235) to (0.0175,0.235) to[out=210,in=0] (-0.075,0.201);}}}
\newcommand{\IOne}{\mathchoice%
	{\FancyOne{\HeightOfTextstyleOne}}%
	{\FancyOne{\HeightOfTextstyleOne}}%
	{\FancyOne{\HeightOfScriptstyleOne}}%
	{\FancyOne{\HeightOfScriptscriptstyleOne}}%
}
\newcommand{\IDigamma}{\tikz[line cap=round,line join=round,line width=0.35]{\draw (0.06,0.2286) to[out=0,in=90] (0.1353,0.172) to (0.1353,0.2286) to (-0.0525,0.2286) to (-0.0415,0.2286) to[out=0,in=90] (-0.0215,0.2086) to (-0.0215,0.02) to[out=270,in=0] (-0.0415,0) to (-0.0525,0) to (0.0605,0) to (0.0415,0) to[out=180,in=270] (0.0215,0.02) to (0.0215,0.2086) to[out=90,in=180] (0.0415,0.2286);\draw (0.025,0.1335) to[out=0,in=90] (0.0968,0.0769) to (0.0968,0.1335) to cycle;}\hspace{0.1ex}\vphantom{\IF}}

\newcommand{\sk}{\operatorname{sk}}
\newcommand\Yo{Y}%{\text{\usefont{U}{min}{m}{n}\symbol{'210}}\hspace{-0.125ex}\vphantom{Y}}
\newcommand{\add}{\mathrm{add}}
\newcommand{\Catst}{\Cat_\infty^\mathrm{st}}
\newcommand{\Verd}{\mathrm{Verd}}
\newcommand{\Kar}{\mathrm{Kar}}

\DeclareFontFamily{U}{min}{}
\DeclareFontShape{U}{min}{m}{n}{<-> udmj30}{}

	
\makeatletter
\renewcommand{\@pnumwidth}{2em} 
\renewcommand{\@tocrmarg}{3em}
\makeatother
%\RedeclareSectionCommand[tocindent+=0.5em]{section}
%\RedeclareSectionCommand[tocindent+=0.5em]{subsection}


\subject{Lecture Notes for}
\title{Algebraic Topology I}
\author{{\normalsize Lecturer}\\
	Stefan Schwede}
\date{{\normalsize Notes typed by}\\
	Michele Lorenzi}
\publishers{Winter Term 2021/22\\
	University of Bonn}

\usepackage{bookmark}
\begin{document}

\setlength{\parindent}{0pt}
\setlength{\parskip}{4pt}

\frontmatter
\KOMAoption{chapterprefix}{false}
\renewcommand{\thedummy}{\arabic{dummy}}
\maketitle
This document contains (unofficial) lecture notes for the course \emph{Algebraic Topology I} given by Prof. Stefan Schwede at Bonn University during the Winter Semester 2021/22.

Everything in these notes should be taken with a grain of salt: I was young and inexperienced when I took them! In particular, the notes were never checked or officially approved by the lecturer, and it goes without saying that I am responsible for any mistakes they may contain. I have a \href{https://github.com/lrnmhl/AT1}{GitHub repository} for the notes, and you are very welcome to use the \href{https://github.com/lrnmhl/AT1/issues}{Issues tab} to report any errors or typos, or make any correction (or you can just tell me in some other way). Note however that major updates are unlikely.

The illustrations were made by \'Alvaro Guti\'errez. Thanks \'Alvaro!

A lot of thanks to (the invaluable) Paul, Yikai, Zhu, for lending me their notes/photos of the blackboard when I was missing stuff, and to Xiaoxiang Zhou and Tzu-Yi Yang (and Peter) for the typo hunting/corrections!

\hrulefill

Last update: \today
	
%Some additions have been made by the author. To distinguish them from the lecture's actual contents, they are labelled with an asterisk. So any \emph{Proof}* or \emph{Lemma}* etc.\ that the reader might encounter are wholly the author's responsibility.
%\\[\thmsep]Please report errors, typos etc.\ through the \href{https://github.com/}{\emph{Issues}} feature of GitHub, or just tell me before or after the lecture.
	
	
\tableofcontents
\listoftoc{lol}
\setcounter{llecture}{0}
\mainmatter\KOMAoption{chapterprefix}{true}
\renewcommand{\thedummy}{\thechapter.\arabic{dummy}}
\renewcommand{\thechapter}{\arabic{chapter}}
\input{Chapters/AT1-Lecture1}
\input{Chapters/AT1-Lecture2}
%%% Lecture 3

\section{The Homotopy Addition Theorem}

\lecture[The Homotopy Addition Theorem: a theorem which is necessary, but a pain to prove.\newline---\emph{\enquote{When homotopy theory was new, people thought this was obvious and didn't feel the need for a proof, until Eilenberg suggested so.}}]{2021-10-18}

This lecture was given by Tobias Lenz, a PhD student of Schwede. I was late to the class, so most of the notes for this lecture are copied from Qi Zhu's notes, thank you Qi Zhu!

\begin{warning}
I took the liberty to rearrange the content of the lecture a little, to improve the exposition. Note that Tobias kindly provided \href{https://uni-bonn.sciebo.de/s/OcE4uI7eVAPzhXg}{his own handwritten notes}.
\end{warning}

\begin{remark}
There's a standing assumption for all of today's lecture: $\pi_k(S^k)\cong\Z$ for all $1\leq k<n$.
\end{remark}

Recall the notion of local degree. Let $f:\ring I^n\to\ring I^n $ be an open embedding, $p\in\ring I^n$. We have that $f$ induces a commutative diagram:
\begin{center}
    \begin{tikzcd}
    H_n(\ring I^n,\ring I^n\sm\cb{p}) \arrow[r, "i_*"] \arrow[d, "f_*" left] \arrow[dr, "f_*"] & H_n(I^n,I^n\sm\cb{p}) & H_n\pair \arrow[l, "i_*" above] \arrow[d, red, dashed, "d\cdot-"] \\
    H_n(f(\ring I^n), f(\ring I^n)\sm\cb{f(p)}) \arrow[r, "i_*" below] & H_n(I^n,I^n\sm\cb{f(p)}) & H_n\pair \arrow[l, "i_*" below]
    \end{tikzcd}
\end{center}
where the maps are all isomorphisms by homotopies and excision, hence they induce the dashed arrow. This is an automorphism of $H_n\pair\cong\Z$ and thus $d=\pm1$. One can show this is independent of $p$, hence we call it the \tbf{local degree} of $f$, and write $\deg(f)=d$.\todo[color=yellow]{There would be more to say about the local degree...}

The main goal of today's lesson is to prove the following theorem.

\begin{theorem}[Homotopy Addition Theorem]\label{theorem:HAT}
Assume we have $\nn{f}{k}:I^n\to I^n$ such that $f_i|_{\ring I^n}$ is an open embedding and the sets $f_i(\ring I^n)$ are pairwise disjoint. Furthermore, let $g:\pair\to (X,A)$ such that $g(I^n\sm\bigcup_{i=1}^k f_i(\ring I^n))\subset A$. Then
\[[g]=\sum_{i=1}^k(\deg f_i)[g\circ f_i]\rightnote{\upshape Here $\deg f_i$ is the local degree!}\]
in $\pi_n(X,A)^\#$.
\end{theorem}

\begin{remark}
Note that we have $f_i(\de I^n)\cap f_j(\ring I^n)=\emptyset$ for all $i,j$.
\end{remark}

\input{Pictures/Lec3Pic1}\label{fig:HAT1}\medskip

\begin{remark}
Two additional remarks about the theorem:
\begin{itemize}
    \item In the early days of homotopy theory people thought this was obvious\rightnote{The statement we gave might appear opaque, but it does seem intuitively true once one thinks about it.} and did not feel the need for a proof, until Eilenberg suggested so.
    \item Tobias: \emph{\enquote{I'm not sure if I can finish the proof today but I was promised an award if I do!}} Spoiler: he could not finish it.
\end{itemize}
\end{remark}

The strategy to prove the homotopy addition theorem is inductive: we want to reduce the problem to the case $k=1$ which is easier, as in that case we can \enquote{blow up} $f_1$ to a map of pairs $(I^n,\de I^n)\to(I^n,\de I^n)$ and we understand well this kind of maps thanks to lemma \ref{lemma:degree-of-sphere-self-maps}. To apply this reduction, we want to \enquote{separate} the images of the $f_i$'s by vertical \enquote{walls}, as in the following illustration.

\input{Pictures/Lec4Pic1}\medskip

This might seem easy but it is in fact the most delicate part of the proof, as the obvious approach, i.e. shrinking the source of the $f_i$'s, could produce maps $gf'_i$ which are not pair homotopic to $gf_i$. To avoid this, we must first modify $g$ into an homotopic map that sends everything outside of a given small subset of the image of every $f_i$ to $A$.

\begin{lemma}\label{lemma:technical-lemma-for-HAT}
For $1\leq i\leq k$, let $\Uu_i\subset f_i(\ring I^n)$ be any non-empty open set. Then $g$ is homotopic relative to $I^n\sm f_i(\mathring{I}^n)$ to a map $g'$ that sends $f_i(I^n)\sm\mathcal{U}_i$ to $A$ for all $i$.\alvaropls\rightnote{\upshape It suffices to check this claim for a single $f_i$, since $f_i(\de I^n)\cap f_j(\ring I^n)$ is empty for all $i,j$.}
\end{lemma}

\begin{remark}
If $[g]=[g']$ in $\pi_n(X,A)^\#$ then $[g\circ f_j]=[g'\circ f_j]$ for all $1\leq j\leq k$.
\end{remark}

\begin{proof}
There is a homotopy relative $\de I^n$ from the identity to a map that sends everything outside of $\ring Q$ to $\de I^n$, where $Q$ is a cube inside $f^{-1}_i(\mathcal{U}_i)$ (basically we take a cube $Q$ inside $f^{-1}_i(\mathcal{U}_i)$ and we blow it to the big cube $\de I^n$ containing $f^{-1}_i(\mathcal{U}_i)$). Composing with $gf_i$ yields a homotopy $H$ of maps of pairs $(I^n,\de I^n)\to(X,A)$ relative $\de I^n$ from $gf_i$ to a map that sends everything outside $\ring Q$ to $A$. We have a map of sets:
\[H':I^n\times I\to X,\quad H'(x,t)=\begin{cases}H(f_i^{-1}(x),t) & x\in f_i(I^n) \\ g(x) & x\in I\sm f_i(\mathring{I}^n)\end{cases}\]
which we can show is well defined. Let $x\in f_i(\de I^n)$, with preimage $y$, then
\[g(x)=H(y,t)=H(y,0)=gf_i(y)\] so that it is well defined as a map of sets. Then $H'(x,0)=g(x)$ and $H'(x,1)\in A$ for $x\not\in f(\mathring{Q})$, in particular $H'(x,1)\in A$ for $x\not\in\mathcal{U}_i$. So $g'=H'(x,1)$ is our candidate map, but it remains to prove that $H'$ is continuous.

Claim. We have a commutative square
\begin{center}
    \begin{tikzcd}[column sep=20mm]
    \de I^n\times I \arrow[d, "i"] \arrow[r, "f_i|_{\de I^n}\times\id"] & (I^n\sm f_i(\ring I^n))\times I \arrow[d, "i"] \\
    I^n\times I \arrow[r, "f_i\times\id"] & I^n\times I
    \end{tikzcd}
\end{center}
Then,
\[(I^n\times I)\amalg_{\de I^n\times I} ((I^n\sm f_i(\mathring{I}^n))\times I)\xto{(f_i\times\id,i)} I^n\times I\]
is a homeomorphism.

\begin{claimproof}
Well-definedness and continuity of the function follow from the universal property of the pushout. One can check that it is bijective by a direct computation. Hence we have a continuous bijection from a quasi-compact space to an Hausdorff space, which is then a homeomorphism. 
\end{claimproof}

Thus to show that $H'$ is continuous, it suffices to show that the maps $H'\circ (f_i\times \id)$ and $H'|_{(I^n\sm f(\ring I^n))\times I}$ are continuous, which follows from the construction.
\end{proof}

\begin{proof}[Proof of the homotopy addition theorem (\ref{theorem:HAT})]\renewcommand{\qedsymbol}{\textit{To be continued...}} Induction on $k$.

$(k=1)$ Take a cube $Z_1\subset f_1(\mathring{I}^n)$. By lemma \ref{lemma:technical-lemma-for-HAT} we may assume that $g(I^n\sm\mathring{Z}_1)\subset A$. Our goal is to construct some $f_1':(I^n,\de I^n)\to(I^n,\de I^n)$ with $[gf_1]=[gf_1']$. There exists\rightnote{The homotopy $Q$ is the same kind of \enquote{pushing out} homotopy that we already considered in the proof of the lemma.} a homotopy $Q$ from $\id_{I^n}$ to a map $p$ such that:
\begin{itemize}
    \item $p|_{Z}=\id_{Z}$, where $Z$ is a cube inside $Z_1$,
    \item $p(I^n\sm\mathring{Z}_1)\subset\de I^n$,
    \item $Q((I^n\sm\mathring{Z}_1)\times I)\subset I^n\sm \mathring{Z}_1$.\alvaropls
\end{itemize}
Then $f_1':= pf_1$ is a map of pairs $(I^n,\de I^n)\to(I^n,\de I^n)$ and we have $f_1\simeq f_1'$ as maps of pairs $(I^n,\de I^n)\to(I^n,I^n\sm \mathring{Z}_1)$, with $Q$ witnessing the homotopy. Moreover, $Qf_1$ is a homotopy of maps of pairs $(I^n,\de I^n)\to(I^n,I^n\sm\mathring{Z}_1)$, so $gQf_1$ will be a homotopy of maps of pairs $(I^n,\de I^n)\to(X,A)$, hence $[gpf_1]=[gf_1]$ in $\pi_n(X,A)^\#$.

To conclude the base case, we want to relate the local degree of $f_1$ (which is the notion involved in the statement of the theorem), to the degree of $f'_1$ as a self map of spheres.

Claim. $\deg f_1$ equals the degree of $f_1'$ as a map $(I^n,\de I^n)\to(I^n,\de I^n)$.

\begin{claimproof}
Pick $x\in f_1^\inv(\ring Z_1)$. Then we have a commutative diagram:
\begin{center}
    \begin{tikzcd}[column sep=small]
    H_n(\mathring{I}^n,\mathring{I}^n\sm\cb{x}) \arrow[r, "i_*",] \arrow[d, "(f_1)_*" left] & H_n(I^n,I^n\sm\cb{x}) \arrow[d, "(f_1)_*"] & H_n(I^n,\de I^n) \arrow[l, "i_*" above] \arrow[d, shift left, "(f_1)_*\ \," left] \arrow[d, shift right, "\ \,(f'_1)_*" right] \arrow[dr, "(f'_1)_*=\,d'\cdot-"]\\
    H_n(\mathring{I}^n, \mathring{I}^n\sm\cb{f_1(x)}) \arrow[r, "i_*"] & H_n(I^n,I^n\sm\cb{f_1(x)}) & H_n(I^n,I^n\sm \mathring{Z}_1) \arrow[l, "i_*" above] & H_n\pair \arrow[l, "i_*"]
    \end{tikzcd}
\end{center}
where the two parallel arrows agree. Each of the horizontal arrows are isomorphisms and multiplication by $\deg f_1$ is another map $H_n(I^n,\de I^n)\to H_n(I^n,\de I^n)$ which makes the diagram commute, hence the degree $d'$ of $f'_1$ must be equal to $\deg f_1$.
\end{claimproof}

Now, if $\deg f_1=1$, then $f_1'\simeq\id$ as maps of pairs, hence $(\deg f_1)[gf_1]=[gf_1']=[g]$.

If instead $\deg f_1=-1$, then $f_1'\simeq r$ where $r$ is reflection in the first coordinate by lemma \ref{lemma:degree-of-sphere-self-maps}, hence $[g]=-[gr]=-[gf_1']=-[gf_1]=(\deg f_1)[gf_1]$.

\end{proof}

%%% Lecture 4

\lecture[We finish the proof of the HAT. My first non-trivial homotopy groups.]{2021-10-25}
$(k\geq2)$ Set $u_i=f_i(\text{center of } I^n)\in I^n$. Assume without loss of generality that the first coordinates of $\nn{u}{k}$ are not all equal (if some of them are, we can \enquote{wiggle} the $f_i$'s).

\input{Pictures/Lec4Pic1}

Choose $t\in (0,1)$ such that
\begin{itemize}
    \item $t$ is different from the first coordinates of $\nn{u}{k}$,
    \item for some $1\leq i\leq k$, the first coordinate of $u_i$ is smaller than $t$,
    \item for some $1\leq i\leq k$, the first coordinate of $u_i$ is larger than $t$.
\end{itemize}

Choose neighborhoods $\Uu_i$ of $u_i$ inside $f_i(\ring I^n)$ that do not intersect $\cb{t}\times I^\ni$, that is such that $\Uu_i$ lies \enquote{on the same side respect to $t$} as $u_i$.

Choose subcubes $Q_i$ inside $\ring I^n$ that contain the center and lie in in $f^{-1}(\Uu_i)$.

Last time we proved that $g$ is pair-homotopic to some $g':\pair\to\pairs$ such that
\[g'(I^n\sm\bigcup_{i=1,\dots,k} f_i(Q_i))\subset A\quad\text{and}\quad[g\circ f_i]=[g'\circ f_i]\text{ in }\pi_n\pairs^\#.\]

We precompose each $f_i:I^n\to I^n$ with the linear shrinking homotopy relative $Q_i$. Set $f'_i=f_i\circ \text{ end of shrinking}$. Then $g'\circ f_i$ is pair homotopic to $g'\circ f'_i$ and $f'_i(I^n)\subset\Uu_i$.

By replacing $g$ by $g'$ and $f_i$ by $f'_i$ we can therefore assume without loss of generality that $f_i(\ring I^n)$ lies on one side of $\cb{t}\times I^\ni$.

\input{Pictures/Lec4Pic2}\smallskip

Write $g=g_1+_t g_2$ by \enquote{cutting along $\cb{x_1=t}$}.

Formally, $g_1(\xs)=g(t\xs)$ and $g_2=g((1-t)x_1+t,\dots,x_n)$.

Set $I_1=\cb{i\in I\mid u_i\text{ lies left of }\cb{x_1=t}}$ and $I_2=\cb{i\in I\mid u_i\text{ lies right of }\cb{x_1=t}}$, where $I=\cb{1,\dots,k}$.

Then by the inductive hypothesis:
\[[g]=[g_1]+[g_2]=\sum_{i\in I_1}\deg(f_i)[g_1\circ f_i]+\sum_{i\in I_2}\deg(f_i)[g_2\circ f_i]=\sum_{i\in I}\deg(f_i)[g\circ f_i].\]\qed

\begin{remark}
Note that we have proved the HAT \textit{assuming} (in order to use lemma \ref{lemma:degree-of-sphere-self-maps}) that $\pi_{n-1}(S^{n-1},*)\cong\Z$, not in full generality! This is enough to prove theorem \ref{theorem:my-first-non-trivial-homotopy-group} which in turn will prove the HAT in each dimension. This is an (awfully confusing) inductive argument!
\end{remark}

\section{My First Non-Trivial Homotopy Group}

\begin{theorem}\label{theorem:my-first-non-trivial-homotopy-group}
Let $n\geq2$ and assume inductively that $\pi_{n-1}(S^{n-1},*)\cong\Z$ so that the HAT holds in dimension $n$. Then $\pi_n(S^n,*)$ is infinite cyclic.
\end{theorem}

\begin{proof}
Choose some point $z\in S^n$. Set $U=S^n\sm\cb{-z}$. Then \[\pi_n(S^n,z)=\pi_n(S^n,\cb{z},z)\underset{U\cong *}{\cong}\pi_n(S^n,U,z)\underset{\pi_1(U,z)=\cb{1}}{=}\pi_n(S^n,U,z)^\dagger\cong\pi_n(S^n,U)^\#,\]
so we may show that $\pi_n(S^n,U)^\#\cong\Z$.

We show that $\pi_n(S^n,U)^\#$ is generated by the class of any pair map $\psi:\pair\to(S^n,U)$ such that $\psi(\de I^n)=\cb{z}$ and $\psi$ factors out a homeomorphism $I^n/\de I^n\cong S^n$.

Let $f:(I^n,\de I^n)\to(S^n,U)$ be any pair map. Set $V=S^n\sm\cb{z}$ so that $S^n=U\cup V$ is an open cover. The Lebesgue Number lemma provides an $m\geq1$ so that each subcube of $I^n$ of side length $1/m$ is mapped by $f$ into $U$ or into $V$. Decompose $I^n$ into $m^n$ subcubes of side length $1/m$. We define subspaces of $I^n$ in this way:
\begin{itemize}[label=-]
    \item $A_{-1}=\de I^n$
    \item $A_0=A_{-1}\cup\cb{\text{\,vertices of the }m^n\text{ subcubes}}$
    \item $A_1=A_0\cup\cb{\text{\,edges of the }m^n\text{ subcubes}}$
    \item $A_2=\cdots$
    \item $A_n=I^n$
\end{itemize}
We want to \enquote{improve} $f$ successively by pair homotopies to maps $f_{-1},f_0,f_1,\dots,f_{n-1}$ such that:
\begin{itemize}
    \item each $f_j$ is homotopic to $f$ relative $\de I^n$,
    \item each $f_j$ is admissible, i.e. it sends every subcube of side length $1/m$ to $U$ or to $V$,
    \item $f_j(A_j)\subset U$ for $j=-1,0,1,\dots,n-1$.
\end{itemize}

We proceed by induction on $j$. For $j=-1$ we can just set $f{-1}=f$. Let now $j\geq 0$. We first modify $f_{j-1}$ and the faces of the $j$-cube. If such a face $Q$ is \enquote{good}, i.e. sent by $f_{j-1}$ into $U$, we do not do anything to $Q$. Otherwise the $(j-1)$-subcube is mapped to $V$ and the restriction of $f_{j-1}$ to it is a pair map $(Q,\de Q)\to (V,V\cap U)$.

Claim. For $j<n$, any pair map $(I^j,\de I^j)\to (V,U\cap V)$ is homotopic relative $\de I^j$ to a map with image in $U\cap V$.

\begin{claimproof}
By stereographic projection $(V,U\cap V)$ is pair homotopic to $(\R^n,\R^n\sm\cb{0})$, hence we can construct a pair map $g:(I^j,\de I^j)\to (\R^n,\R^n\sm\cb{0})$. Because $\de I^j$ is compact and $0\notin f(\de I^j)$, there is an $\epsilon>0$ such that $g(\de I^j)\cap(\epsilon\text{-ball around }0)=\emptyset$.

So $g:(I^j,\de I^j)\to(\R^n,\R^n\sm\ring B(\epsilon,0))$. Now, $\R^n$ can be obtained from $\R^n\sm\ring B(\epsilon,0)$ by attaching an $n$-cell. The cellular approximation theorem (for CW-pairs) and the fact that $I^j$ is a $j$-dimensional CW-complex gives us a relative homotopy from $g$ to a cellular map. Since $j<n$, the cellular map has image in $\R^n\sm\ring B(\epsilon,0)\subset\R^n\sm\cb{0}$.
\end{claimproof}

We can now change $f_{j-1}|A_j$ into $f_j|A_j$ by a homotopy relative $A_{j-1}$ into a map that sends all $j$-cells to $U$.

We use the HEP for $(I^n,A_j)$ to extend $f_j$ to all of $I^n$; we can repeatedly use the HEP with target $U$ or with target $V$ to ensure that the map $f_j$ is again admissible.

After this inductive construction we can replace $f$ by $f_{n-1}$ and we have arranged without loss of generality that $f(A_{n-1})\subset U$.

To sum up: we can now assume that any $g:\pair\to(S^n,U)$ satisfies $g(A_{n-1})\subset U$ and each top-dimensional subcube is mapped to $U$ or to $V$.

We apply the HAT to the map $g$ with $\nn{f}{k}$ the reparametrizations of those subcubes that are \textit{not} mapped into $U$ (and hence into $V$). Then by the HAT we have:
\[[g]=\sum_i\pm[g\circ f_i]\text{ in }\pi_n(S^n,U)^\#\cong\pi_n(S^n,\cb{z},z)^\dagger\cong\pi_n(S^n,z)\]

With our choice of the representative $g$, we have gained that each summand on the right hand side is in the image of the homomorphism $\pi_n(V,V\cap U)^\#\to\pi_n(S^n,U)^\#$, which is then surjective.\rightnote{It might be worth to point this out: clearly $\pi_n(S^n,*)$ cannot be finite, because there exist self-maps of the $n$-sphere of every integer degree, and degree is an homotopy invariant.} By the long exact homotopy group sequence of the pair $(V,V\cap U)$, we obtain
\[\pi_n(V,V\cap U,*)\cong\pi_n(\R^n,\R^n\sm\cb{0},*)\cong\pi_\ni(\R^n\sm\cb{0},*)\cong\pi_\ni(S^\ni,*)\cong\Z,\]
so $\pi_n(S^n,U)^\#\cong\pi_n(S^n,z)$ is infinite cyclic.
\end{proof}

\section{Reminder on Simplicial Sets}\label{section:reminder-on-sset}

Our proof of Hurewicz theorem will make substantial use of the theory of simplicial sets, therefore we remind some notions before going on.

We denote by $\Delta$ the category with objects the sets $[n]=\cb{0,1,\dots,n}$ for $n\geq 0$ and morphisms the non-decreasing maps.

A \textbf{simplicial set} is a contravariant functor from $\Delta$ to sets, $X:\Delta^\op \to\Set$. The set of the $n$-simplices is denoted $X_n=X([n])$, for a morphism $\alpha:[n]\to[m]$ write $\alpha^*=X(\alpha):X_m\to X_n$.

The \textbf{singular complex} (singular simplicial set) of a space $Y$ is the simplicial set
\[\S(Y):\Delta^\op \to\Set, \quad [n] \mapsto \S(Y)_n :=\left\{\text{continuous maps }f:\nabla^n \to Y\right\},\]
where $\nabla^n=\{(\xso)\in\R^{n+1}\mid x_i\geq0,\ \sum x_i=1\}$.

For $\alpha: [n]\to [m]$, the map $\alpha^*:\S(Y)_m\to\S(Y)_n$ is precomposition with the affine linear map $\alpha_*:\nabla^n\to\nabla^m$ defined by $e_i\mapsto e_{\alpha(i)}$, i.e. the map
\[(\nno{t}{m})\mapsto\alpha_*(t)_j=\sum_{j=\alpha(i)}t_j.\]

For a continuous map $\psi:Y\to Z$, a morphism of simplicial sets $\psi_*:=\S(\psi):\S(Y)\to\S(Z)$ is given by $\S(\psi)_n(f)=\psi\circ f$. This yields a functor $\S:\Top\to\sSet$.

The \textbf{geometric realization} is a functor $|-|:\sSet\to\Top$ defined as follows. For a simplicial set $X$ we set
\[|X|=\left(\coprod X_n\times\nabla^n\right)/\sim\]
where $X_n$ is endowed with the discrete topology and the equivalence relation is the one generated by:\normalmarginpar\marginnote{\footnotesize The equivalence relation is only \textit{generated} by the condition stated, which is not symmetric. The actual equivalence relation is not easy to understand. We will return on this problem with the theory of \hyperref[subsection:minimal-representatives]{minimal representatives}.}
\[X_m\times\nabla^m\cont(x,\alpha_*(t))\sim(\alpha^*(x),t)\in X_n\times\nabla^n\quad\text{for all }\alpha:[n]\to[m],\ x\in X_m,\ t\in\nabla^n.\]

\reversemarginpar

\input{Chapters/AT1-Lecture5}
\input{Chapters/AT1-Lecture6}
\input{Chapters/AT1-Lecture7}
\input{Chapters/AT1-Lecture8}
%%% Lecture 9

\section{More on Fibre Bundles and Fibrations}

\lecture[Some more fibre bundles/fibrations stuff. Prof. Schwede suggests that we take the categorical red pill. Shout-out to the category of compactly generated spaces (without definition). We introduce the compact-open topology.]{2021-11-15}

Prof. Schwede is back!

\begin{theorem}
Let $p:E\to B$ be a continuous map, with path-connected base.
\begin{numerate}
    \item If $p$ is a Hurewicz fibration, then any two fibers of $p$ are homotopy equivalent.
    \item If $p$ is a Serre fibration, then any two fibres which are CW-complexes are homotopy equivalent.
\end{numerate}
\end{theorem}

\begin{proof}
Let $b_0,b_1\in B$ be points, set $F_0=p^{-1}(b_0),F_1=p^{-1}(b_1)$.

Let $\omega:[0,1]\to B$ be a path from $b_0$ to $b_1$. Choose homotopies $H$ and $K$ as liftings in the followings diagrams:
\begin{center}
\begin{tikzcd}[column sep=large]
F_0\times 0 \arrow[r,"\incl"] \arrow[d,hook] & E \arrow[d,"p"] \\
F_0\times[0,1] \arrow[ur,dashed,"H"] \arrow[r,"\omega\circ\pr_2"] & B
\end{tikzcd}\qquad
\begin{tikzcd}[column sep=large]
F_1\times 0 \arrow[r,"\incl"] \arrow[d,hook] & E \arrow[d,"p"] \\
F_1\times[0,1] \arrow[ur,dashed,"K"] \arrow[r,"\bar\omega\circ\pr_2"] & B
\end{tikzcd}
\end{center}

where $\bar\omega(t)=\omega(1-t)$ is the inverse path.

We set $f=H(-,1):F_0\to F_1$ and $g=K(-,1):F_1\to F_0$.

Let $L:[0,1]\times[0,1]\to B$ be a homotopy, relative $\cb{0,1}$, from the concatenated path $\omega*\bar\omega$ to $\const_{b_0}$.
\[L(-,0)=\omega*\bar\omega,\]
\[L(-,1)=\const_{b_0}\]
\[L(0,t)=L(1,t)=b_0\]

Choose another lifting in the following diagram:

\begin{center}
    \begin{tikzcd}[column sep=huge]
    F_0\times((0\times I)\cup(I\times 0)\cup(1\times I)) \arrow[r] \arrow[d,hook] & E \arrow[d,"p"] \\

    F_0\times[0,1]\times[0,1] \arrow[ur,dashed,swap,"\bar L"] \arrow[r,swap,"L\circ\pr_{2,3}"] & B
    \end{tikzcd}
\end{center}
where the upper arrow is\alvaropls\ $(\const_{\incl}\,\cup\, H*(K\circ(f\times\id))\,\cup\,\const_{\incl\circ 
g\circ f})$.\smallskip

\input{Pictures/Lec9Pic1}

\vspace{-0.1cm}
Then we can set $G=\bar L(-,-,1):F_0\times[0,1]\to E$ to obtain a homotopy from $\incl:F_0\into E$ to $\incl\circ\,g\circ f:F_0\to E$.

Since $p\circ G=\const_{b_0}$, this homotopy $G$ takes place inside $F_0$. So $G$ can be seen as a homotopy from $\id_{F_0}$ to $g\circ f$. Reversing the roles of $b_0$ with $b_1$, $F_0$ with $F_1$ and $f$ with $g$ yields a homotopy from $\id_{F_1}$ to $f\circ g$.
\end{proof}

\unnumpar{Induced fibres bundles/fibrations}

We construct the pullback bundle\rightnote{This construction is also called base change sometimes. Or fiber product.}. Let $p:E\to B$ and $\beta:B'\to B$ be continuous maps. The pullback is \[E'=B'\times_B E=\cb{(b',e)\in B'\times E\mid \beta(b')=p(e)}\]
with subspace topology of the product topology.

\begin{center}
    \begin{tikzcd}
    &[-6ex] (b',e) \arrow[r,mapsto] & e \\[-4ex]
    (b',e) \arrow[d,mapsto] & B'\times_B E \arrow[d,dashed,"p'"] \arrow[r,dashed,"\beta'"] & E \arrow[d,"p"] \\
    
    b' & B' \arrow[r,"\beta"] & B
    \end{tikzcd}
\end{center}

This is a pullback in $\Top$ in the sense of category theory, i.e. the following universal property holds: for all spaces $A$ and continuous maps $\alpha:A\to B'$ and $\epsilon:A\to E$ such that $\beta\alpha=p\epsilon$, there is a unique continuous map $\delta:A\to B'\times_B E$ such that $\beta'\delta=\epsilon$ and $\alpha\delta=p'$, namely the map $\delta(a)=(\alpha(a),\epsilon(a))$.
\begin{center}
    \begin{tikzcd}
    A \arrow[dr,dashed,"\exists!\delta"] \arrow[drr,bend left,"\epsilon"] \arrow[ddr,bend right,"\alpha"] & & \\
     & B'\times_B E \arrow[d,"p'"] \arrow[r,"\beta'"] & E \arrow[d,"p"] \\
    
     & B' \arrow[r,"\beta"] & B
    \end{tikzcd}
\end{center}

\begin{example}[Restriction bundle]
Suppose $B'$ is a subspace of $B$ and $\beta:B'\to B$ the inclusion. Then we have
\[B'\times_B E\cong p^{-1}(B')\]
with homeomorphisms given by
\[(b',e)\mapsto e,\]
\[(p(e),e)\mapsfrom e.\]
\end{example}

\begin{example}[A pretty stupid one]
I lost the explanation, but take it as a little rebus:
\begin{center}
    \begin{tikzcd}
    B'\amalg B' \arrow[d] \arrow[r] & B' \arrow[d] \\
    *\amalg* \arrow[r] & *
    \end{tikzcd}
\end{center}
\end{example}
In general taking the constant map $\beta:B'\to B$ to a point $b\in B$ will yield as the pullback bundle the trivial bundle $B'\times p^{-1}(b)$.

\begin{theorem}
Let $p:E\to B$ and $\beta:B'\to B$ be continuous maps.
\begin{itemize}
    \item[(i)] If $p$ is a fibre bundle, then so is $p':B'\times_B E\to B'$.
    \item[(ii)] If $p$ is a Hurewicz fibration, then so is $p'$.
    \item[(iii)] If $p$ is a Serre fibration, then so is $p'$.
\end{itemize}
\end{theorem}

\begin{proof}\rightnote{\emph{\enquote{You take the blue pill — the story ends, you wake up in your bed and believe whatever you want to believe. You take the red pill — you stay in Wonderland, and I show you how deep the rabbit hole goes.}} (cringe, sorry)}There's two reasonable proofs of the first fact, one direct, one categorical. The Professor seems to be suggesting a blue pill/red pill situation (with the categorical version being \emph{\enquote{much more transparent}} to him).

(i) Consider any point $a\in B'$. There is a open neighbourhood $U$ of $\beta(a)$ in $B$ and a local trivialization, i.e. a homeomorphism $\psi:p^{-1}(U)\to U\times F$ for one space $F$ (over $U$).

We argue that $p':B'\times_B E\to B'$ is trivializable over $V=\beta^{-1}(U)$ with the same fibre $F$.

The following are mutually inverse homeomorphisms:
\begin{align*}
    (p')^{-1}(V)&\xleftrightarrow{\ \cong\ }V\times F\\
    (b',e)&\ \,\mapsto\ (b',\psi_2(e))\\
    (b',\psi^{-1}(\beta(b'),f))&\ \,\mapsfrom\ (b',f).
\end{align*}

Categorical proof: the statement is an instance of the fact that \enquote{pullbacks are transitive}.

In any category $\Cc$, we consider a commutative diagram:
\begin{center}
    \begin{tikzcd}[column sep={8em,between origins}]
    B''\times_B E \arrow[d] \arrow[r] & B'\times_B E \arrow[d,"p'"] \arrow[r,"\beta'"] & E \arrow[d] \\
    
    B'' \arrow[r] & B' \arrow[r,"\beta"] & B
    \end{tikzcd}
\end{center}

If both squares are pullbacks, then the composite square is also a pullback. Symbolic notation:
\[B''\times_B E\cong B''\times_{B'}(B'\times_B E).\]

Hence, considering the same situation as in the direct proof, we have:
\[(p')^{-1}(V)\cong V\times_B E\cong V\times_{B'}(B'\times_B E)\cong V\times_U(U\times_B E)\cong V\times_U(U\times F)\cong V\times F.\]

(ii)+(iii) We show: whenever $p:E\to B$ has the HPL for some space $X$, then $p':B'\times_B E\to B'$ also has the HLP for $X$.

Consider a lifting square on the left:
\begin{center}
    \begin{tikzcd}[column sep=large,row sep=huge]
    X\times0 \arrow[d,hook] \arrow[r,"f"] & B'\times_B E \arrow[d,"p'" near end] \arrow[r,"\beta'"] & E \arrow[d,"p"] \\
    
    X\times[0,1] \arrow[ur,dashed,bend left=20,blue,"\til H"] \arrow[urr,dashed,crossing over,red, "K" near start] \arrow[r,"H"'] & B' \arrow[r,"\beta"'] & B
    \end{tikzcd}
\end{center}

There is a homotopy $K:X\times[0,1]\to E$ so that $p\circ K=\beta\circ H$ and $K(-,0)=\beta'\circ f$.

The universal property of the pullback provides a unique continuous map
\[\til H:X\times[0,1]\to B'\times_B E\]
such that $p'\circ \til H=H$ and $\beta'\circ\til H=K$.

The two continuous maps $f,\,\til H(-,0):X\times0\to B'\times_B E$ compose in the same way with $p'$ and with $\beta'$. The uniqueness part of the universal property then forces $f=\til H(-,0)$.
\end{proof}

\chapter{Mapping Spaces}

\section{Compact-open Topology}

Our aim: to define and study a specific topology on $Z^X=\cb{f:X\to Z\text{ continuous}}.$

We would like the "exponential law" to hold: the map
\begin{align*}
    Z^{X\times Y}&\to(Z^X)^Y\\
    (f:X\times Y\to Z)&\mapsto \cb{y\mapsto f(-,y)}
\end{align*}
should be a homeomorphism.

Unfortunately, it is not. At least not in general, but the good news is that it is whenever $Y$ is Hausdorff and $X$ locally compact.

\begin{remark}
There is a way to arrange the exponential law in complete generality: work in the full subcategory $CG$ of compactly generated spaces. Then $CG$ is a cartesian closed category, i.e. it has finite products and for all $X\in\operatorname{ob}(G)$
\[-\times X:CG\to CG\]
has a right adjoint, written $Z\mapsto Z^X$.

Watch out: the product in $CG$ is not always the usual product topology!

Some examples of classes of spaces in $CG$ are:
\begin{itemize}[label={-}]
    \item every locally compact Hausdorff space,
    \item every CW-complex,
    \item every manifold is in $CG$,
    \item the realization of every simplicial set,
    \item for two CW-complexes $X$ and $Y$, the product in $CG$, $X\times_{CG}Y$, is again a CW-complex.
    \item for all simplicial sets $A,B$, the canonical map:
    \[|A\times B|\to|A|\times_{CG}|B|\]
    is a homeomorphism.
\end{itemize}
\end{remark}

Enter: the compact-open topology. For spaces $X$ and $Z$, write $Z^X$ for the set of continuous maps $f:X\to Z$. Let $K$ be a compact subset of $X$ and let $O$ be an open subset of $Z$. Set:
\[W(K,O)=\cb{f\in Z^X:f(K)\subset O}\subset Z^X.\]

The \textbf{compact-open topology} on $Z^X$ is the topology generated by the sets $W(K,O)$ with $K\subset X$ compact and $O\subset Z$ open, i.e. these sets form a subbasis of the compact-open topology.

\begin{theorem}
Let $X$ be a compact space and $(Z,d)$ a metric space.
\begin{numerate}
\item There is a metric on $Z^X$, defined as:
\[d(g_1,g_2)=\sup_{x\in X}d(g_1(x),g_2(x)),\text{ for }g_1,g_2\in Z^X.\]
called the supremum metric.
\item The compact-open topology on $Z^X$ coincides with the metric topology of the supremum metric.
\end{numerate}
\end{theorem}

\begin{proof}
(1) Omitted (see \cite[proposition A.13]{hatcher}).

(2) \enquote{$\subset$}: compact-open is metrically open.

It suffices to show that the generating sets $W(K,O)$ are metrically open. Fix $K\subset X$ compact, $O\subset Z$ open and $f\in W(K,O)$, i.e. $f(K)\subset O$. Then $C=Z\sm O$ is a closed subset of $Z$ and $f(x)\not\in C$ for all $x\in K$ (hence $d(f(x),C)>0$ for all $x\in K$). Since $K$ is compact, $\epsilon=\inf_{x\in K}d(f(x),C)>0$
is positive.
We claim that $B_{\sup}(f,\epsilon)\subset W(K,O)$.

Let $g\in Z^X$ be such that $d(f,g)<\epsilon$. Then $d(f(x),g(x))<\epsilon$ for all $x\in K$, hence:
\[d(f(x),g(x))\leq d(f(x),C),\]
so $g(x)\in O=Z\sm C$. Therefore $g(K)\subset O$, so $g\in W(K,O)$.

\enquote{$\supset$}: metrically open is compact-open.

Let $A\subset Z^X$ be open in the sup-topology, $f\in A$. We will construct finitely many compact sets $K_i$ in $X$ and open sets $O_i$ in $Z$ so that:
\[f\in W(K_1,O_1)\cap\dots\cap W(K_m,O_m)\subset A.\]

Since $A$ is open in the sup-topology, there is an $\epsilon>0$ so that $A$ contains the $\epsilon$-ball around $f$.

For $x\in X$ the set $f^{-1}(B(f(x),\epsilon/5))$ is an open neighbourhood of $x$ in $X$. Since $X$ is compact\rightnote{Note: compact spaces, meaning quasi-compact \textit{and} Hausdorff, are locally compact, but quasi-compact spaces may not be!}, this contains a compact neighbourhood $K_x$ of $x$. Since $X$ is compact, there are finitely many $x_1,\dots,x_m$ such that $X=K_{x_1}\cup\dots\cup K_{x_m}$.
Set $K_i:=K_{x_i}$ and $O_i:=$ open $\epsilon/5$-neighbourhood of $f(K_i)$ in $Z$.

\input{Pictures/Lec9Pic2}
As shown in the picture, for all $z,z'\in O_i$, we have $d(z,z')\leq 4\epsilon/5<\epsilon$ (to expand on this, observe we have $f(K_x) \subset B(f(x),\frac{\varepsilon}{5})\subseteq O_x := \frac{\varepsilon}{5}\text{-neighborhood of }f(K_x)$, hence both $z$ and $z'$ are at distance less or equal than $\frac{\varepsilon}{5}$ from two points of $f(K_x)$, and both these points are at distance less or equal than $\frac{\varepsilon}{5}$ from $f(x)$).

Suppose that $g\in W(K_1,O_1)\cap\dots\cap W(K_m,O_m)$. Then for all $x\in X$ there is $1\leq i\leq m$ with $x\in K_i$. Since $g\in W(K_i,O_i)$, we have $g(x)\in O_i$. We also have $x\in K_i=K_{x_i}$, so $d(f(x),f(x_i))\leq\epsilon/5$ so $f(x)\in O_i$. Hence $d(g(x),f(x))<\epsilon$. Since this holds for all $x\in X$, $d(f,g)<\epsilon$. So $W(K_1,O_1)\cap\dots\cap W(K_m,O_m)\subset B_{\sup}(f,\epsilon)\subset A$.
\end{proof}

\input{Chapters/AT1-Lecture10}
% Lecture 11

\lecture[We continue studying mapping spaces.]{2021-11-22}

\tit{Proof.}\rightnote{\Attention\ I definitely still need to digest lecture 11 and 12, don't count on everything I write. Update: I think I have digested the lectures, but still don't count on everything I write.}
First, note that $q^*$ is injective, continuous and has the desired image by the universal property of the quotient topology. We show that $q^*$ is also open as a map onto its image. We let $K\subset Y$ be compact and $O\subset Z$ be open. Then
\[q^*(W(K,O))=\cb{g\circ q:X\to Z\mid g\in Z^Y,\ g(K)\subset O}=W(q^{-1}(K),O)\cap\im(q^*).\]
Because $K$ is compact and $Y$ Hausdorff, $K$ is closed in $Y$; so $q^{-1}(K)$ is closed in $X$. Since $X$ is compact, $q^{-1}(K)$ is again compact. So $W(q^{-1}(K),O)\cap\im(q^*)$ is open.\qed

\begin{example}
The map $q:[0,1]\to S^1=\cb{z\in\CC:|z|=1}$, $q(x)=e^{2\pi ix}$ is a quotient map, hence
\[q^*:X^{S^1}\to\cb{f\in X^{[0,1]}\mid f(0)=f(1)}\]
is a homeomorphism. For any base point $x_0\in X$, this homeomorphism restricts to a homeomorphism
\[(X,x_0)^{(S^1,1)}\to\Omega X.\]

If $(Y,y_0)$ is a pointed space, the pointed version of $[Y,Z]\onto\pi_0(Z^Y)$ yields a well-defined surjective map $[Y,Z]_*\to\pi_0((Z,z_0)^{(Y,y_0)})$. If $Y$ is locally compact, this is a bijection.

For $Y=S^1$, this gives a bijection $\pi_1(Z,z_0)\cong\pi_0((Z,z_0)^{(S^1,1)})=\pi_0(\Omega Z)$.
\end{example}

\subsection{Loop Shifts the Homotopy Groups}

Let $(Y,y_0)$ be a based space. The reduced suspension is the space:
\[\Sigma Y=\frac{Y\times[0,1]}{Y\times 0\cup\cb{y_0}\times[0,1]\cup Y\times 1}.\]

The quotient map $q:Y\times[0,1]\to\Sigma Y$ induces a continuous injection:
\[q^*:(Z,z_0)^{(\Sigma Y,*)}\to Z^{Y\times[0,1]}\]
whose image consists of all continuous maps $f:Y\times[0,1]\to Z$ that factor through the quotient map, i.e. such that $f(Y\times 0\cup\cb{y_0}\times[0,1]\cup Y\times 1)=\cb{z_0}$. If $Y$ is compact, then so are $Y\times[0,1]$ and $\Sigma Y$, and $q^*$ is even an homeomorphism onto its image (by lemma \ref{lemma:quotient-map-induces-homeomorphism-on-mapping-spaces}), hence (using theorem \ref{theorem:exponential-law}):
\[(Z,z_0)^{(\Sigma Y,y_0)}\cong(\Omega Z,\const_{z_0})^{(Y,y_0)}.\]

In particular,
\[\Hom_{\Top_*}((\Sigma Y,*),(Z,z_0))\cong\Hom_{\Top_*}((Y,y_0),(\Omega Z,\const_{z_0})).\]
so the functors $\Sigma$ and $\Omega$ are adjoint endofunctors in the category of based spaces.

On path components, we obtain a bijection
\[[\Sigma Y,Z]_*\cong[Y,\Omega Z]_*.\]

For $Y=S^n$ the last bijection specializes to a bijection
\[\pi_{n+1}(Z,z_0)=[S^{n+1},Z]_*\cong[\Sigma S^n,Z]_*\cong[S^n,\Omega Z]_*=\pi_n(\Omega Z,\const_{z_0})\]
which is even a group homomorphism (as we will see).

\section{Mapping Spaces and Serre Fibrations}

\begin{theorem}
Let $Z$ be any space, $(X,A)$ a relative CW-complex with $X$ and $A$ finite\rightnote{\upshape A more general statement could be easily obtained by requiring just that $X$ (and hence $A$) be locally compact. The statement we give is tailored to our purposes.} CW-complexes. Then the restriction map $\incl^*:Z^X\to Z^A$, $f\mapsto f|_A$ is a Serre fibration.
\end{theorem}

\begin{proof}
We show that $\incl^*$ has the HLP with respect to all CW-complexes $Q$. So we consider a lifting diagram
\[\begin{tikzcd}
Q\times0 \arrow[d,hook] \arrow[r,"f"] & Z^X \arrow[d,"\incl^*"] \\
Q\times[0,1] \arrow[r,"\Phi"] & Z^A
\end{tikzcd}\]
Since $X$ and $A$ are locally compact we can use the exponential law to adjoint $f$ and $\Phi$ to continuous maps
\[\til f:X\times Q\times0\to Z\]
\[\til\Phi:A\times Q\times[0,1]\to Z\]
The condition $\incl^*\circ f=\Phi|_{Q\times0}$ becomes the relation that $\til f$ and $\til\Phi$ coincide on $A\times Q\times0$.

We consider the glued map
\[\til f\cup\til\Phi:X\times Q\times 0\cup A\times Q\times[0,1]\to Z.\]

Since $(X\times Q,A\times Q)$ is a CW-pair, it has the HEP, so there is a continuous extension $H:X\times Q\times[0,1]\to Z$ that extends $\til f$ and $\til\Phi$. The adjoint $H^\#:Q\times[0,1]\to Z^X$ of $H$ then solves the original lifting problem.
\end{proof}

Applied to the relative CW-complex $([0,1],\cb{0,1})$ the theorem says that the restriction map
\[Z^{[0,1]}\to Z^{\cb{0,1}}\cong Z\times Z,\ w\mapsto (w(0),w(1))\]
is a Serre fibration for every space $Z$.

We let $z_0\in Z$ be any base point. Define $EZ$ as the subspace of $Z^{[0,1]}$ of all paths that start at $z_0$. Equivalently, $EZ$ is the pullback:
\begin{center}
    \begin{tikzcd}
    w \ar[d,mapsto] &[-6ex] EZ \arrow[d] \arrow[r,hook] & Z^{[0,1]} \arrow[d] &[-6ex] w \ar[d,mapsto] \\
    w(1) & Z \arrow[r,"{(z_0,-)}"] & Z\times Z & (w(0),w(1))
    \end{tikzcd}
\end{center}

Since Serre fibrations are stable under base-change, we conclude that the map $EZ\to Z$, $w\mapsto w(1)$ is a Serre fibration.

\begin{theorem}\label{theorem:EZ-contractible}
The space $EZ$ is contractible onto the constant path at $z_0$.
\end{theorem}

\begin{proof}
Let $H:[0,1]\times[0,1]\to[0,1]$ be a homotopy, relative $\cb{0}$, that contracts the interval onto $0$, e.g. $H(s,t)=s(1-t)$. Let
\[H^*:Z^{[0,1]}\to Z^{[0,1]\times[0,1]}\]
be the continuous induced map and
\[\til H^*:Z^{[0,1]}\times[0,1]\to Z^{[0,1]}\]
its adjoint, i.e. $\til H^*(w,t)(s)=w(H(s,t))$.

We observe that
\[\til H^*(w,t)(0)=w(H(0,t))=w(0).\]
So whenever $w(0)=z_0$ (i.e. $w\in EZ$), then also $\til H^*(w,t)(0)=z_0$. In other words, for all $t\in[0,1]$, the map $\til H^*(-,t)$ takes $EZ$ to $EZ$. So we can restrict $\til H^*$ to a continuous map $\til H^*:EZ\times[0,1]\to EZ$, which is the desired contracting homotopy. Indeed we have
\[\til H^*(w,1)(s)=w(H(s,1))=w(0),\]
so the homotopy $\til H^*$ ends in the constant map at $z_0$, and for all $w\in EZ$
\[\til H^*(w,0)(s)=w(H(s,0))=w(s),\]
so $\til H^*(w,0)=w$.
\end{proof} 

Now we can see that $\pi_n(\Omega Z,*)\cong\pi_{n+1}(Z,z_0)$ is a group isomorphism.

\begin{proof}[Second proof/construction of the isomorphism $\pi_n(\Omega Z,*)\cong\pi_{n+1}(Z,z_0)$]

We have seen that the space $EZ=\cb{w\in Z^{[0,1]}\mid w(0)=z_0)}$ is contractible.

The map $e:EZ\to Z$, $e(w)=w(1)$ is a Serre fibration, so for every point in $Z$, we get a long exact sequence of homotopy groups. For $z_0\in Z$, the fibre of $e$ at $z_0$ is $\Omega Z$; hence the long exact sequence of homotopy groups of $\Omega Z\to EZ\to Z$ gives:
\[\dots\to\pi_{n+1}(EZ,*)=0\xto{e_*}\pi_{n+1}(Z,z_0)\xto{\de}\pi_n(\Omega Z,*)\xto{\incl_*}\pi_n(EZ,*)=0\to\cdots\]

So for $n\geq 1$, the connecting morphism $\de:\pi_{n+1}(Z,z_0)\to\pi_n(\Omega Z,*)$ is an isomorphism.
\end{proof}

\section{Turning Maps into Fibrations, up to Homotopy}

The constructions in this section, $Ef$ and the homotopy fibre $\ho_{y_0}$, are somehow dual to the mapping cylinder and the mapping cone, as we will see.

\begin{theorem}\label{theorem:factorization-equivalence-fibration}
Every continuous map $f:X\to Y$ can be factored functorially and naturally as a composite
\[X\xto{\cong}Ef\xto{p} Y\]
of a homotopy equivalence followed by a Serre fibration.
\end{theorem}

\begin{proof}
Let $Ef=X\times_Y Y^{[0,1]}=\cb{(x,w)\in X\times Y^{[0,1]}\mid f(x)=w(0)}$. More precisely, $Ef$ is the pullback:
\[\begin{tikzcd}
Ef \arrow[d] \arrow[r] & Y^{[0,1]} \arrow[d] &[-6ex] w \ar[d,mapsto] \\
X \arrow[r,"f"] & Y & w(0)
\end{tikzcd}\]

We define natural continuous maps $h:X\to Ef$ by $h(x)=(x,\const_{f(x)})$ and $p:Ef\to Y$ by $p(x,w)=w(1)$.

Clearly we have $f=p\circ h$.

We observe that $Ef$ can be described as a slightly different pullback:
\[\begin{tikzcd}
(x,w) \ar[d,mapsto] &[-6ex] Ef \arrow[d] \arrow[r] & Y^{[0,1]} \arrow[d] &[-6ex] w \ar[d,mapsto]\\
(x,w(1)) & X\times Y \arrow[r,"f\times\id"] & Y\times Y & (w(0),w(1))
\end{tikzcd}\]

Since Serre fibrations are stable under base change, the map $Ef\to X\times Y$, $(x,w)\mapsto(x,w(1))$ is a Serre fibration. Since $\pr_2:X\times Y\to Y$ is a Serre fibration and Serre fibrations are closed under composition, $p$ is a Serre fibration. 

We are left to prove that the map $h:X\to Ef$ is a homotopy equivalence. We show that the homotopy inverse, which is also a left inverse, is the projection to the first factor.

Claim. The composite $c:Ef\to Ef$ is homotopic to the identity.

\begin{claimproof}
We define the desired homotopy
\[\bar H:Ef\times[0,1]\to Ef=X\times_Y Y^{[0,1]}\]
by specifying its projections to $X$ and to $Y^{[0,1]}$. The first coordinate of $\bar H$ is $Ef\times[0,1]\to X$, $(x,w,t)\mapsto x$, i.e. the constant homotopy of the projection to $X$. The second coordinate is the composite
\[Ef\times[0,1]\xto{\pr_2\times\id} Y^{[0,1]}\times [0,1]\xto{\til H^*} Y^{[0,1]}\]
where $\til H^*$ is the homotopy we constructed in the proof of theorem \ref{theorem:EZ-contractible}.

This has the following properties:
\begin{itemize}[label={-}]
    \item $\bar H$ starts with the identity,
    \item for all $t\in [0,1]$, all $w\in Y^{[0,1]}$, the path $\til H^*(w,t)$ has the same startpoint as $w$, so $\bar H$ really lands in $Ef$.
    \item $\til H^*(w,1)$ is constant at $w(0)$, which means that $\bar H(-,1)=h\circ\pr_1$.
\end{itemize}
\end{claimproof}
\end{proof}

\input{Chapters/AT1-Lecture12}
\input{Chapters/AT1-Lecture13}
% Lecture 14

\lecture[Existence of EM-spaces (btw, today is supposedly Dies Academicus...).]{2021-12-1}

We can now prove the existence of EM-spaces.\rightnote{I missed this lecture because I thought on the Dies Academicus there were no lectures. Apparently the Dies Academicus starts at 10, though. -.-}

\begin{theorem}
Let $n\geq2$ and let $A$ be an abelian group. Then there is an EM-space of type $(A,n)$ that is a CW-complex.
\end{theorem}

\begin{proof}
We choose a free resolution of $A$ as an abelian group:
\[0\to\Z[I]\xto{d}\Z[J]\xto{\epsilon}A\to0,\]
i.e. a short exact sequence of abelian groups with $I,J$ some sets.

We define a CW-complex with skeleta:
\[X^{(0)}=\cdots=X^{(\ni)}=\cb{x},\]
\[X^{(n)}=\cb{x}\cup_{J\times S^\ni}J\times D^n\cong\bigvee_J S^n.\]
By cellular approximation $X^{(n)}$ is $(n-1)$-connected. The Hurewicz theorem then provides an isomorphism
\[h:\pi_n(X^{(n)},x)\xto{\cong}H_n(X^{(n)};\Z)\cong\Z[J].\]

For every index $i\in I$ we chose a map $\alpha_i:S^n\to X^{(n)}$ such that $h([\alpha_i])=d(i)\in\Z[J]$. We define
\[X^{(n+1)}=X^{(n)}\cup_{I\times S^n}I\times D^{n+1}\]
using the $\alpha_i$'s as attaching maps. Then $X^{(n+1)}$ is a CW-complex with one $0$-cell, an $n$-cell for every element of $J$ and an $(n+1)$-cell for every element of $I$. So the cellular chain complex of $X^{(n+1)}$ is concentrated in degrees $0$, $n$ and $n+1$ and in the relevant dimensions it looks as follows:
{\small
\[0\to C^\text{cell}_{n+1}(X^{(n+1)};\Z)=\underbrace{H_{n+1}(X^{(n+1)},X^{(n)};\Z)}_{\cong\Z[I]}\xto{\de}C^\text{cell}_n(X^{(n+1)};\Z)=\underbrace{H_n(X^{(n)},\cb{x};\Z)}_{\cong\Z[J]}\to0.\]}Hence we have $H_n^\text{cell}(X^{(n+1)};\Z)=\coker(\de)\cong\coker(d:\Z[I]\to\Z[J])\cong A$.

$X^{(n+1)}$ is again $(n-1)$-connected, so the Hurewicz theorem provides an isomorphism
\[\pi_n(X^{(n+1)},x)\xto{\cong}H_n(X^{(n+1)};\Z)\cong H^\text{cell}_n(X^{(n+1)};\Z)\cong A.\]

Now we can use kill the homotopy groups as we have seen in theorem \ref{theorem:killing-homotopy-groups}, obtaining a relative CW-complex $(X,X^{(n+1)})$ with relative cells in dimensions $n+2$ and higher and such that:
\[\pi_i(X,x)\cong\pi_i(X^{(n+1)},x)\cong\begin{cases}
0 & \text{for } 1\le i < n\\
A & \text{for } i = n
\end{cases}\]
and $\pi_i(X,x)=0$ for $i\ge n+1$. So $X$ is a CW-complex and a $K(A,n)$.
\end{proof}

There is also an alternative construction of $K(A,n)$ as the geometric realization of some simplicial set.\todo{I am missing this part!}

\section{Uniqueness of EM-Spaces}

We first prove an auxiliary lemma.

\begin{lemma}\label{theorem:extension-theorem}
Let $(X,Y)$ be a relative CW-complex, $Z$ any space. Suppose that for all $m\geq1$ such that $(X,Y)$ has at least one relative $m$-cell, $\pi_{m-1}(Z,z)=0$ holds for all $z\in Z$. Then every continuous map $f:Y\to Z$ has a continuous extension to $X$.
\end{lemma}

\begin{proof}
We construct inductively continuous maps $f^{(n)}:X^{(n)}\to Z$ on the relative skeleta, that successively extend each other. Then $g=\cup f^{(n)}:X=\cup X^{(n)}\to Z$ does the job.\rightnote{Just to be sure:\\ do not forget that maps constructed like this are continuous thanks to CW-complexes having the final topology with respect to the inclusions of\\ their skeleta.}

We start with $f=f^{(-1)}:X^{(-1)}=Y\to Z$. We extend $f^{(-1)}$ to $f^{(0)}:X^{(0)}=Y\amalg J_0\to Z$ by mapping the relative $0$-cells arbitrarily to $Z$.

Let now $m\ge1$ and suppose that $f^{(m-1)}$ has been constructed. If there are no relative $m$-cells, we set $f^{(m)}=f^{(m-1)}$. Otherwise choose characteristic maps for the relative $m$-cells, call $\alpha_j$ the attaching maps and consider the following composite:
\[S^{m-1}\xto{\alpha_j}X^{(m-1)}\xto{ f^{(m-1)}}Z.\]
Pick a basepoint $x\in S^{m-1}$. Then $f^{(m-1)}\circ\alpha_j$ represents an element in $\pi_{m-1}(Z,z)$, with $z=f^{(m-1)}(\alpha_j(x))$. By hypothesis we have $\pi_{m-1}(Z,z)=0$, hence the $f^{(m-1)}\circ\alpha_j$ all represent the trivial class, i.e. they can be extended to the disk $D^m$, which gives us a way to extend $f^{(m-1)}$ to a map $f^{(m)}$ defined on $X^{(m)}$.
\end{proof}

\begin{theorem}\label{theorem:single-0-cell-complex}
Every $(\ni)$-connected CW-complex $Y$ is homotopy equivalent to a CW-complex whose $(\ni)$-skeleton is one $0$-cell.
\end{theorem}

\begin{proof}
Suppose that $n=1$. We can choose a maximal tree\rightnote{Strictly speaking, we would have to show that such a maximal tree exists, but heh.} in $Y$ (i.e. a 1-dimensional subcomplex containing all the $0$-cells) and collapse it to obtain the desired CW-complex with just one cell (the quotient map will be an homotopy equivalence).

Now suppose that $n\ge2$. If $Y$ is $(n-1)$-connected, its $n$-skeleton is still $(n-1)$-connected. Since the Hurewicz map is an isomorphism, considering
\[\pi_n(Y^{(n)},y)\to H_n(Y^{(n)};\Z)\cong H^\text{cell}_n(Y^{(n)};\Z)=\ker(\underbrace{C^\text{cell}_n(Y^{(n)};\Z)}_{\text{free abelian}}\xto{d^\text{cell}}C^\text{cell}_\ni(Y^{(n)};\Z))\]
we have that $H_n(Y^{(n)};\Z)$ and hence $\pi_n(Y^{(n)},y)$ are free abelian groups.

We choose a basis $I$ of the free abelian group $\pi_n(Y^{(n)},y)$ and represent the basis elements by continuous maps $\alpha_i:S^n\to Y^{(n)}$. These $\alpha_i$'s together define a map
\[\alpha=\bigvee_{i\in I}\alpha_i:\bigvee_I S^n\to Y^{(n)}\]
with source and target $(n-1)$-connected CW-complexes. The map $\alpha$ induces an isomorphism on $H_n(-;\Z)$ because it sends the natural basis of $\vee_I S^n$ to the chosen basis of $H_n(Y^{(n)};\Z)$. Note also that source and target of $\alpha$ have trivial homology above dimension $n$. Hence $\alpha$ is a homology isomorphism between simply-connected CW-complexes, hence a homotopy equivalence.

By cellular approximation we can assume that $\alpha$ is cellular for the CW-structure on $\vee_I S^n$ with one $0$-cell and a $n$-cell for every $i\in I$. Then we form
\[Y'=Y\cup_{Y^{(n)}}\bigvee_I S^n\]
where the gluing is along a cellular homotopy inverse $g:Y^{(n)}\to \vee_I S^n$. The space $Y'$ then comes with a CW-structure with one $0$-cell and no cells in dimensions $1,\dots,\ni$ and the map
\[Y\to Y\cup_{Y^{(n)}}\bigvee_I S^n=Y'\]
is a homotopy equivalence (using the homotopy extension property).
\end{proof}

\input{Chapters/AT1-Lecture15}
% Lecture 16

\lecture[The core of the representability argument.]{2021-12-07}

Given the above discussion, for all CW-complexes $X$, the set $[X,K(A,n)]$ becomes an abelian group by $[f]+[g]=[\mu\circ(f,g)]$.

\begin{remark}
One can realize a $K(A,n)$ as a topological abelian group, e.g. $|\til A[\Delta^n/\de]|$. $\til A[\Delta^n/\de]$ is a simplicial abelian group, and $|-|:\sSet\to\Top_\text{cpt. gen.}$ commutes with products, which implies that $|\til A[\Delta^n/\de]|$ is an abelian topological group with product given by:
\[|\til A[\Delta^n/\de]|^2\xto{\cong}|\til A[\Delta^n/\de]\times\til A[\Delta^n/\de]|\xto{|\mu|}|\til A[\Delta^n/\de]|.\]
\end{remark}

\begin{example}
We know that $\rp{\infty}=K(\Z/2,1)$ and $\cp{\infty}=K(\Z,2)$. We also talked about how $\rp{\infty}$ classifies principal $\Z/2$-bundles (i.e. two-fold coverings). Then for every paracompact space $X$, we have:
\[
\begin{tikzcd}
    \left[X,\rp{\infty}\right] \ar[r,"\cong"] & \operatorname{Prin}_{\Z/2}(X) \ar[r,"\cong"] & \Pic_\R(X)\\[-4ex]
    f \ar[rr,mapsto] & & f^* (j)\\[-4ex]
    & E\to X \ar[r,mapsto] & E\times_{C_2}\R\to X\\[-4ex]
    & (F-s_0(X))/\R_{>0}\to X & F\to X \ar[l,mapsto]
\end{tikzcd}
\]
where $j$ is the tautological line bundle on $\rp{\infty}$ and $s_0$ is the zero section. Under this bijection, the group operation on the left corresponds to the tensor product on the right (?).

Similarly we have:
\[[X,\cp{\infty}]\xto{\cong}\Pic_\CC(X)\]
sending $[f:X\to\cp{\infty}]$ to pullback of the universal complex line bundle over $\cp{\infty}$ and again with addition corresponding to the tensor product of line bundles (?).
\end{example}

\subsection{The Fundamental Cohomology Class}

\begin{propdef}[Fundamental cohomology class]\label{propdef:fundamental-cohomology-class}
Let $(X,\phi)$ be an EM-space of type $(A,n)$, $n\ge1$, $A$ abelian. Then there is a unique class $\iota\in H^n(X;A)$ such that the composite:
\[\pi_n(X,x)\xto{h}H_n(X;\Z)\xto{\Phi(\iota)}A\]
is the isomorphism $\phi:\pi_n(X,x)\to A$, where $\Phi:H^n(X;A)\to\Hom(H_n(X;\Z),A)$ is the homomorphism from the universal coefficient theorem.
\end{propdef}

\begin{proof}
Since $X$ is $(\ni)$-connected, the Hurewicz homomorphism $h:\pi_n(X,x)\to H_n(X,\Z)$ is an isomorphism if $n\ge2$; for $n=1$ this is true because $A$ is abelian. Since $H_\ni(X,\Z)$ is trivial (for $n\ge2$) or free (if $n=1$), the $\Ext$-term in the UCT vanishes, so
\[\Phi:H^n(X;A)\xto{\cong}\Hom(H_n(X;\Z),A)\]
is an isomorphism. Then we can define $\iota=\Phi^\inv(\phi\circ h^\inv)$.
\end{proof}

We want to see that for a CW-complex $Y$ the fundamental class $\iota\in H^n(K(A,n);A)$ gives rise to a natural isomorphism:
\[
[Y,K(A,n)]\to H^n(Y;A),\quad [f:Y\to K(A,n)]\mapsto f^*(\iota).
\]

The convention for $n=0$ is that $K(A,0)$ is just the group $A$ with the discrete topology, the map $\mu:K(A,0)^2\to K(A,0)$ is the addition on $A$ and $i: K(A,0)\to K(A,0)$ is the inverse map, while $\iota\in H^0(A,A)$ is represented by the identity cocycle.

\begin{theorem}
For all $n\ge0$ and all abelian groups $A$, the evaluation map at $\iota$ is a group morphism.
\end{theorem}

\begin{proof}
Given the convention above, without loss of generality we can assume $n\ge1$.

We start with a \enquote{universal example}, from which the general case will easily follow. Let $Y=K(A,n)\times K(A,n)$ and $p_1,p_2:K(A,n)^2\to K(A,n)$ the projections. Observe that the sum $[p_1]+[p_2]$ is represented by the map $\mu$ realizing group addition:
\[
K(A,n)\times K(A,n)\xto{(p_1,p_2)=\id}K(A,n)\times K(A,n)\xto{\mu}K(A,n),\]
\[\text{i.e. } [p_1]+[p_2]=[\mu\circ(p_1,p_2)]=[\mu].
\]\ 

Then we want to show that in the special case of $[\mu]=[p_1]+[p_2]$ evaluation at $\iota$ is additive:
\[\mu^*(\iota)=p_1^*(\iota)+p_2^*(\iota).\]
We will use the following isomorphism:
{\small
\[ H^n(K(A,n)^2;A)\xto{\cong}\Hom(H_n(K(A,n)^2;\Z),A)\xto{\cong}\Hom(\pi_n(K(A,n)^2,*),A)\xto{\cong}\Hom(A\times A,A)\]}where the first isomorphism comes from the UCT (and K\"unneth theorem) and the second is precomposition with the Hurewicz isomorphism. Under this isomorphism $\mu^*(\iota)$ corresponds (as one can show by naturality of the first two maps) to the addition $A\times A\to A$ and the projections to the corresponding projections $A\times A\to A$. But in $\Hom(A\times A,A)$ clearly we do have that the sum of the projections is the addition, hence the same holds in $H^n(K(A,n)^2;A)$.

Now for the general case, let $Y$ be an arbitrary space and $f,g:Y\to K(A,n)$ continuous maps. Then we have:
\begin{align*}
(f+g)^*(\iota)=(\mu\circ(f,g))^*(\iota)=(f,g)^*(\mu^*(\iota))&=(f,g)^*(p_1^*(\iota)+p_2^*(\iota))\\
&=(f,g)^*(p_1^*(\iota))+(f,g)^*(p_2^*(\iota))\\
&=f^*(\iota)+g^*(\iota).
\end{align*}
\end{proof}

\begin{lemma}
For all $n\ge1$, all abelian groups $A$ and all based CW-complexes $Y$, the forgetful map
\[[Y,K(A,n)]_*\to [Y,K(A,n)]\]
is a bijection.
\end{lemma}

\begin{proof}
We start by showing surjectivity. We will use that for $n\ge1$ any $K(A,n)$ is path-connected and that the inclusion of the basepoint $\cb{y}\into Y$ has the HEP\rightnote{Note (for those who studied the HEP long ago): clearly not every inclusion of a point has the HEP, this works (as always in this course) because $Y$ is a CW-complex.}. In particular, given any continuous map $f:Y\to K(A,n)$, we can choose a path $w:[0,1]\to K(A,n)$ from $f(y)$ to $x\in K(A,n)$. The HEP for $(Y,\cb{y})$ lets us choose a homotopy $H:Y\times[0,1]\to K(A,n)$ starting with $f$ and such that $H(y,-)=w$. Then $g=H(-,1)$ is freely homotopic to $f$ and based, hence the forgetful map sends $[g]$ to $[f]$.

To show injectivity, we first consider the case $n=1$. Let $f,g:Y\to K(A,1)$ be based maps that are freely homotopic. Then we have:
\[f_*=g_*:H_1(Y;\Z)\to H_1(K(A,1);\Z).\]
Since $A$ is abelian, the map $\pi_1(K(A,1),x)\to H_1(K(A,1),\Z)$ is an isomorphism, hence we have:
\[f_*=g_*:\pi_1(Y,y)\to \pi_1(K(A,1),x).\]
By theorem \ref{theorem:correspondence-maps-and-group-morphisms}\leftnote{In principle, theorem \ref{theorem:correspondence-maps-and-group-morphisms} would require $Y$ to be connected, but (Xiaoxiang Zhou suggested) it is easy to show that when $Y$ is not connected the result still holds: the idea is that a pointed map from a disjoint union\\ of connected components can\\ be decomposed as maps from each component, only one of which has to preserve the basepoint.}, $f$ and $g$ are then based homotopic.

Now let $n\ge2$. We will exploit the fact that a $K(A,n)$ is then simply connected. Let $Z$ be (more generally) any simply-connected space, $f,g:Y\to Z$ two based continuous maps that are freely homotopic. Let $H:Y\times[0,1]\to Z$ be a free homotopy from $f$ to $g$. Then $w=H(y,-):[0,1]\to Z$ is a loop at the basepoint of $Z$. Since $Z$ is simply-connected, this loop is homotopic to the constant loop at the endpoint, relative to it, say by a homotopy $G:[0,1]\times[0,1]\to Z$. The HEP of the pair $(Y\times[0,1],Y\times\cb{0}\cup\cb{y}\times[0,1]\cup Y\times\cb{1})$ applied to
\[K:(Y\times\cb{0}\cup\cb{y}\times[0,1]\cup Y\times\cb{1})\times[0,1]\xto{\const_f\cup\,G\,\cup\, \const g} Z\]
yields a map $\bar K:Y\times[0,1]\times[0,1]\to Z$ extending $K$ and such that $\bar K(-,-,0)=H$. Then the map $H'=\bar K(-,-,1):Y\times[0,1]\to Z$ is a based homotopy between $f$ and $g$.
\end{proof}

We are now ready to prove the result we promised.

\begin{theorem}
For all $n\ge0$, all abelian groups $A$ and all CW-complexes $Y$, the evaluation map
\[
    [Y,K(A,n)]\to H^n(Y;A),\ [f]\mapsto f^*(\iota)
\]
is a group isomorphism.
\end{theorem}

\begin{proof}\renewcommand{\qedsymbol}{\textit{To be continued...}}
If $n=0$, we defined $K(A,0)$ as $A$ with the discrete topology, so clearly
\[[Y,K(A,n)]=\Hom_\Set(\pi_0(Y),A)\cong H^0(Y;A)\]
with addition in $[Y,K(A,n)]$ corresponding to pointwise addition in $\Hom_\Set(\pi_0(Y),A)$.

Now let $n\ge1$. The theorem holds for $Y=\emptyset$ (since both sides are then $0$), so without loss of generality we can assume $Y$ is non-empty. We choose a basepoint $y\in Y$. We will start by proving the theorem in a special case, which will then be used to conclude in generality.

Special case. Suppose $Y$ is $(n-1)$-connected. Then $H_\ni(Y;\Z)$ is trivial (if $n\ge2$, by Hurewicz) or at least free (if $n=1$), hence $\Ext(H_\ni(Y;\Z),A)$ vanishes and the morphism $\Phi:H^n(Y;A)\to\Hom(H_n(Y;\Z),A)$ from the UCT is an isomorphism. We also have that for $n\ge2$ the Hurewicz map $h:\pi_n(Y,y)\to H_n(Y;\Z)$ is an isomorphism, while for $n=1$ it is the universal homomorphism into the abelianization. In both cases precomposition yields an isomorphism
\[\Hom(h,A):\Hom_\Grp(H_n(Y;\Z),A)\xto{\cong}\Hom_\Grp(\pi_n(Y,y),A).\]
The composite
\[[Y,K(A,n)]_*\xto{[f]\mapsto f^*(\iota)} H^n(Y;A)\underset{\cong}{\xto{\Phi}}\Hom(H_n(Y;\Z),A)\underset{\cong}{\xto{\Hom(h,A)}}\Hom_\Grp(\pi_n(Y,y),A)\]
sends $[f]$ to $\pi_n(f):\pi_n(Y,y)\to\pi_n(K(A,n),*)\to A$, hence we know that it is a bijection by theorem \ref{theorem:correspondence-maps-and-group-morphisms} (since $Y$ is a $(n-1)$-connected CW-complex and $K(A,n)$ has vanishing homotopy groups for $k>n$). Since the composite, the second and the third maps are bijections, the valuation at $\iota$ map $[f]\mapsto f^*(\iota)$ must also be.
\end{proof}

% Lecture 17

\lecture[We finish to prove representability of singular cohomology. We show existence and uniqueness up to homotopy of CW-approximations to a topological space $Z$.]{2021-12-14}

\comment{
Last time we introduced the fundamental class of an EM-space $\iota\in H^n(K(A,n);A)$, the unique class such that the composite:
\[\pi_n(K(A,n),*)\xto{\cong} H_n(K(A,n);\Z)\xto{\Phi(\iota)}A\]
where the last morphism comes from the UCT, is the identification $\phi:\pi_n(K(A,n),*)\xto{\cong} A$.

For every CW-complex $X$, the map
\[[X,K(A,n)]\to H^n(X;A)\]
\[[f]\to f^*(\iota)\]
is a group homomorphism and the forgetful map $[X,K(A,n)]_*\to[X,K(A,n)]$ is bijective.

\begin{theorem}
For all $n\ge0$, all abelian groups $A$ and all CW-complexes $Y$, the evaluation homomorphism $[Y,K(A,n)]\to H^n(Y;A)$ is an isomorphism.
\end{theorem}
}

General case. We consider the cone of the $(\ni)$-skeleton, \[CY^{(\ni)}=(Y^{(\ni)}\times[0,1])/(Y^{(\ni)}\times\cb{1}),\]
and we form the CW-complex
\[Y\cup_{Y^{(\ni)}}CY^{(\ni)}.\]
Note that we have continuous maps
\[Y\xto{i} Y\cup_{Y^{(\ni)}}CY^{(\ni)}\xto{p}\Sigma Y^{(\ni)},\]
where $i$ is the inclusion and $p$ collapses $Y$.\todo[color=red]{Here $\Sigma Y$ denotes the \tit{unreduced} suspension! (I think, I do not know why the Professor chose this notation)}

We consider the commutative diagram of abelian groups:
\[
\begin{tikzcd}\label{diagram:representability-theorem}
{[\Sigma Y^{(\ni)},K(A,n)]} \ar[d] \ar[r,"p^*"] & {[Y\cup_{Y^{(\ni)}}CY^{(\ni)},K(A,n)]} \ar[d] \ar[r, "i^*"] & {[Y,K(A,n)]} \ar[d,"?"] \ar[r] & 0\\
H^n(\Sigma Y^{(\ni)};A) \ar[r,"p^*"] & H^n(Y\cup_{Y^{(\ni)}}CY^{(\ni)};A) \ar[r,"i^*"] & H^n(Y;A) \ar[r] & 0
\end{tikzcd}\tag{$*$}
\]
We want to use the $5$-lemma to show that the map on the right is an isomorphism.

Exactness of the upper row of (\ref{diagram:representability-theorem}). In the relative CW-complex $(Y\cup_{Y^{(\ni)}}CY^{(\ni)},Y)$ all relative cells have dimension less or equal to $n$. Since the homotopy groups of $K(A,n)$ vanish up to dimension $\ni$, every continuous map $Y\to K(A,n)$ admits a continuous extension to $Y\cup_{Y^{(\ni)}}CY^{(\ni)}$. So the upper map $i^*$ is surjective. Since $p\circ i$ is the constant map, we have that $i^*\circ p^*=(p\circ i)^*$ is the zero homomorphism. Let $f:Y\cup_{Y^{(\ni)}}CY^{(\ni)}\to K(A,n)$ represent an element in the kernel of $i^*$, i.e. $f|_Y$ is nullhomotopic. The HEP for the pair $(Y\cup_{Y^{(\ni)}}CY^{(\ni)},Y)$ let us replace $f$ by a homotopic map $g:Y\cup_{Y^{(\ni)}}CY^{(\ni)}\to K(A,n)$ such that $g|_Y$ is the constant map at the basepoint:
\[
\begin{tikzcd}
Y \ar[dr,"\const_*"'] \ar[r,"i"] & Y\cup_{Y^{(\ni)}}CY^{(\ni)} \ar[d,"g"] \ar[r,"p"] & \Sigma Y^{(\ni)} \ar[dl,dashed,"h"]\\
& K(A,n) &
\end{tikzcd}
\]
So there is a unique continuous map $h:\Sigma Y^{(\ni)}\to K(A,n)$ with $h\circ p=g$. In particular, we have that $p^*[h]=[g]=[f]$, so that $\ker(i^*)=\im(p^*)$.

Exactness of the lower row of (\ref{diagram:representability-theorem}). This comes from the cohomology long exact sequence of the relative CW-complex $(Y\cup_{Y^{(\ni)}}CY^{(\ni)},Y)$. In particular, observe that the sequence would continue with
\[H^n(Y;A)\xto{\de} H^{n+1}(\Sigma Y^{(\ni)};A)\]
and $H^{n+1}(\Sigma Y^{(\ni)};A)=0$, since $\Sigma Y^{(\ni)}$ is a CW-complex of dimension less or equal to $n$.

The left vertical map in (\ref{diagram:representability-theorem}) is surjective. Since $\Sigma K(A,\ni)$ has trivial homotopy groups below the $n$-th and $K(A,n)$ above the $n$-th, using theorem \ref{theorem:correspondence-maps-and-group-morphisms} we can choose a continuous based map, unique up to homotopy, $\kappa_n:\Sigma K(A,\ni)\to K(A,n)$, that realizes the following homomorphism on homotopy groups:
\[
    \pi_n(\Sigma K(A,\ni),*)\underset{\text{Hurewicz}}{\xto{\cong}}H_n(\Sigma K(A,\ni);\Z)\underset{\text{Suspension}}{\xto{\cong}} H_\ni(K(A,\ni);\Z)
\]
\[
    \underset{\text{Hurewicz}}{\xto{\cong}}\pi_\ni(K(A,\ni),*)\xto{\cong}A\xto{\cong}\pi_n(K(A,n),*).
\]

Then the following square commutes:
\[
\begin{tikzcd}[column sep=10em]
{[Y^{(\ni)},K(A,\ni)]} \ar[d,"\cong","{[f]}\mapsto f^*(\iota_\ni)"'] \ar[r,"{[f]}\mapsto{[\kappa_n\circ\Sigma f]}"] & {[\Sigma Y^{(\ni)},K(A,n)]} \ar[d,"{[f]}\mapsto f^*(\iota_n)"]\\
H^\ni(Y^{(\ni)};A) \ar[r,"\cong"] & H^n(\Sigma Y^{(\ni)};A)
\end{tikzcd}
\]
which gives us surjectivity of the map ${[f]}\mapsto f^*(\iota_n)$ (i.e. the left vertical map in (\ref{diagram:representability-theorem})), since ${[f]}\mapsto f^*(\iota_\ni)$ is an isomorphism by induction.

The middle vertical map in (\ref{diagram:representability-theorem}) is an isomorphism. This follows from the previous special case, since the space $Y\cup_{Y^{(\ni)}}CY^{(\ni)}$ is $(\ni)$-connected: indeed, by cellular approximation any continuous map $f:S^k\to Y\cup_{Y^{(\ni)}}CY^{(\ni)}$, with $k\le\ni$, is homotopic to a map with image the contractible space $CY^{(\ni)}$.

The $5$-lemma then shows that the right vertical map in (\ref{diagram:representability-theorem}) is an isomorphism.\qed

\begin{example}
$\rp{\infty}$ is both an EM-space of type ($\Z/2,1$) and a classifying space for real line bundles. So for any CW-complex $X$ we get two natural bijections
\[
    \Pic_\R(X)\xleftarrow{\cong}[X,\rp{\infty}]\xto{\cong}H^1(X,\FF_2),
\]
given by:
\[
    f^*(j)\mapsfrom [f]\mapsto f^*(\iota)
\]
where $j$ is the universal/tautological line bundle on $\rp{\infty}$. We can combine the two bijections into a map $w_1:\Pic_\R(X)\to H^1(X;\FF_2)$ which is called the first \tbf{Stiefel-Whitney class}.

One can show that a real line bundle is completely determined (up to isomorphism) by its first Stiefel-Whitney class (it is possible to see that vector bundles of higher rank are in general \tit{not} determined by their Stiefel-Whitney classes, though).

The complex version of this story uses $\cp{\infty}$:
\[
    \Pic_\CC(X)\xleftarrow{\cong}[X,\cp{\infty}]\xto{\cong} H^2(X,\Z),
\]
this gives a map $\Pic_\CC(X)\to H^2(X;\Z)$ which is called the first \tbf{Chern class}. Again, one can show that a complex line bundle over a CW-complex is determined up to isomorphism by its first Chern class (and that the same is not true in general for vector bundles of higher rank and their Chern classes).
\end{example}

\section{CW Approximation}

Our aim: we want to show that every space $Z$ admits a weak homotopy equivalence $X\to Z$ from a CW-complex $X$ that is unique up to homotopy.

A continuous map $f:X\to Y$ is a \tbf{weak homotopy equivalence}\rightnote{Being \enquote{weakly homotopy equivalent} is the equivalence relation generated by the relation \enquote{there is a weak homotopy equivalence between $X$ and $Y$}, which can be proved transitive but is not symmetric!} if $\pi_0(f):\pi_0(X)\to\pi_0(Y)$ is bijective and for all $n\ge1$ and all $x\in X$, $\pi_n(f):\pi_n(X,x)\to\pi_n(Y,f(x))$ is an isomorphism.\todo{A bunch of interesting facts are in AT1Sheet10.2 and elsewhere...}

Note that every homotopy equivalence is a weak homotopy equivalence (note: sometimes people (e.g. me) forget that this is not entirely trivial... see \cite[proposition 1.18]{hatcher} for a proof in the case of fundamental groups, which can be easily generalized to the higher homotopy groups).

For an example of a weak homotopy equivalence which is not an homotopy equivalence one could take the inclusion of a point into the long line or the Warsaw circle, or a map from a countable discrete space $Y$ to $\cb{0}\cup\cb{1/n\mid n\ge1}\subset\R$ with the subspace topology.\todo[color=yellow]{These pathological counterexamples can be quite subtle (if maybe not very interesting?).}

\begin{theorem}[Existence of CW-approximations]\label{theorem:cw-approximation}
Let $Z$ be a path connected space, $X$ a CW-complex, $x_0\in X$ a $0$-cell, $f:(X,x_0)\to(Z,z_0)$ a continuous map. Then there is a CW-complex $Y$ that contains $X$ as a subcomplex and a continuous extension $g:Y\to Z$ that is a weak homotopy equivalence.

As a special case, $X=\cb{x_0}$ gives the existence of CW-approximations.
\end{theorem}

\begin{proof}
This is similar to the proof of \ref{theorem:killing-homotopy-groups}, the method for "killing homotopy groups". We construct inductively CW-complexes $X=Y^{(-1)}\subset Y^{(0)}\subset Y^{(1)}\subset\cdots$ such that $Y^{(n)}$ contains $Y^{(\ni)}$ as a subcomplex and a continuous map $g^{(n)}:Y^{(n)}\to Z$ such that $g^{(n)}|_{Y^{(\ni)}}=g^{(\ni)}$ and $\pi_k(g^{(n)})=0$ for all $1\le k\le n$. Then $Y=\cup_{n\ge0}Y^{(n)}$ with the weak topology is the desired CW-complex and $g=\cup_{n\ge0}\,g^{(n)}:Y\to Z$ is the desired map because $\pi_k(g)=0$\rightnote{We are appealing to the fact that a compact subspace of a CW complex is contained in a finite subcomplex. In case of doubt: \cite[Prop. A.1]{hatcher}.} for all $k\ge1$ by compactness, hence $g$ is a weak homotopy equivalence by its associated long exact homotopy sequence.

Now, we proceed with the construction. Let $Y^{(-1)}=X$ and $g^{(-1)}=f:X\to Z$. Suppose that $Y^{(\ni)}$ and $g^{(\ni)}:Y^{(\ni)}\to Z$ with the desired properties have already been constructed. For each class $i\in\pi_n(g^{(\ni)})$ choose representing maps $\alpha_i:S^\ni\to Y^{(\ni)}$ and $\beta_i:D^n\to Z$ such that the diagram
\[
\begin{tikzcd}
S^\ni \ar[d,hook] \ar[r,"\alpha_i"] & Y^{(\ni)} \ar[d,"g^{(\ni)}"]\\
D^n \ar[r,"\beta_i"] & Z
\end{tikzcd}
\]
commutes. By cellular approximation we can assume that the $\alpha_i$ are cellular maps.

We construct $Y^{(n)}$ by attaching $n$-cells to $Y^{(\ni)}$
using the $\alpha_i$'s as attaching maps. Similarly, we define $g^{(n)}$ taking the union of $g^{(\ni)}$ with the $\beta_i$'s. Then $Y^{(n)}$ contains $Y^{(\ni)}$ as a subcomplex and $g^{(n)}$ extends $g^{(\ni)}$.

It remains to show that $\pi_k(g^{(n)})=0$ for $1\le k\le n$. To this end we compare the long exact homotopy sequences of $g^{(\ni)}$ and $g^{(n)}$:
{\small
\[
\begin{tikzcd}[scale=0.5]
\pi_k(Y^{(\ni)},x_0) \ar[d,"\incl_*"] \ar[r,"g^{(\ni)}_*"] & \pi_k(Z,z_0) \ar[d,eq] \ar[r] & \pi_k(g^{(\ni)}) \ar[d,"\incl_*"] \ar[r,"\de"] & \pi_{k-1}(Y^{(\ni)},x_0) \ar[d,"\incl_*"] \ar[r,"g^{(\ni)}_*"] & \pi_{k-1}(Z,z_0) \ar[d,eq]\\
\pi_k(Y^{(n)},x_0) \ar[r,"g^{(n)}_*"] & \pi_k(Z,z_0) \ar[r] & \pi_k(g^{(n)}) \ar[r,"\de"] & \pi_{k-1}(Y^{(n)},x_0) \ar[r,"g^{(n)}_*"] & \pi_{k-1}(Z,z_0)
\end{tikzcd}
\]}
The fourth map is surjective for $k\le n$ by cellular approximation, hence the middle map is surjective by the $5$-lemma. Since $\pi_k(g^{(\ni)})=0$ for $1\le k<n$, we have that also $\pi_k(g^{(n)})=0$ for $1\le k<n$.

For $k=n$ the map $\incl_*:\pi_n(g^{(\ni)})\to\pi_n(g^{(n)})$ sends all elements to zero by design: given an $i\in\pi_n(g^{(\ni)})$ represented by $(\beta_i,\alpha_i)$, we have the diagram
\[
\begin{tikzcd}[row sep=large]
S^\ni \ar[d,hook] \ar[r,"\alpha_i"] & Y^{(\ni)} \ar[d,"g^{(\ni)}"' near start] \ar[r,hook] & Y^{(n)} \ar[d,"g^{(n)}"]\\
D^n \ar[r,"\beta_i"'] \ar[urr,dashed,crossing over] & Z \ar[r,eq] & Z
\end{tikzcd}
\]
where the dashed arrow is a characteristic map for the $i$-th cell. The existence of the diagonal filler means that the outer square represents the $0$ element in $\pi_n(g^{(n)})$\leftnote{Because then you can contract $D^n$.}, hence $\incl_*$ is surjective and the zero homomorphism, so $\pi_n(g^{(n)})=0$.
\end{proof}

\begin{theorem}[Uniqueness of CW-approximations]\label{theorem:uniqueness-cw-approximations}
Let $f:X\to Z$ and $g:Y\to Z$ be two CW-approximations, i.e. weak homotopy equivalences to a path-connected space $Z$ such that $X$ and $Y$ are CW-complexes. Then there is a homotopy equivalence $\psi:X\to Y$ such that the diagram:
\[
\begin{tikzcd}
X \ar[rd,"f"'] \ar[rr,"\psi"] & & Y \ar[dl,"g"]\\
& Z &
\end{tikzcd}
\]
commutes up to homotopy.
\end{theorem}

\begin{proof}
We consider the map $f\amalg g:X\amalg Y\to Z$. Then by the previous theorem (\ref{theorem:cw-approximation}) there is a CW-complex $W$ that contains $X\amalg Y$ as a subcomplex and a weak homotopy equivalence $\Phi:W\to Z$ with $\Phi|_X=f$ and $\Phi|_Y=g$.

Because $g$ and $\Phi$ are weak homotopy equivalences, so is the inclusion $i_Y:Y\into W$
\[
\begin{tikzcd}
Y \ar[rd,"g"'] \ar[rr,"i_Y"] & & W \ar[dl,"\Phi"]\\
& Z &
\end{tikzcd}
\]
Since $Y$ and $W$ are path-connected, $i_Y$ is a homotopy equivalence by Whitehead's theorem. Let $j:W\to Y$ be a homotopy inverse.

Interchanging the roles of $X$ and $Y$ shows that also $i_X:X\to W$ is a homotopy equivalence.
So $\psi:j\circ i_X:X\to Y$ is a homotopy equivalence. Moreover: \[g\circ\psi=g\circ j\circ i_X=\Phi\circ i_Y\circ j\circ i_X\simeq\Phi\circ i_X=f.\]
\end{proof}

% Lecture 18

\lecture[We apply the results on representability of singular cohomology.]{2021-12-15}

Addendum to last time: the results on CW-approximations hold for all spaces $Z$, not necessarily path-connected. Indeed, a space $Z$ is the union of its path components, hence by choosing CW-approximations for each path component separately and taking the disjoint union we obtain a CW-approximation for $Z$ (note: we might get some strange CW-complex).

\section{Some Applications: Classifying Cohomology Operations}

\begin{theorem}
Let $f:X\to Y$ be a weak homotopy equivalence. Then $f$ induces isomorphisms $f_*:H_n(X;A)\to H_n(Y;A)$ and $f^*:H^n(Y;A)\to H^n(X;A)$ for all abelian groups $A$.
\end{theorem}

\begin{proof}
By the UCTs, it suffices to prove the homological case for $A=\Z$.

Because the simplices $\ns$ are all path-connected, $H_n(X;A)$ decomposes as the direct sum of the homology of the path components:
\[\bigoplus_{i\in\pi_0(X)}H_n(X_i;A)\to H_n(X;A)\]
and similarly for $Y$. Since $f_*:\pi_0(X)\to\pi_0(Y)$ is bijective, it suffices to show the claim for each $X_i$. So we can assume without loss of generality that $X$ and $Y$ are path-connected.

We let $Z(f)=X\times[0,1]\cup_f Y$ be the mapping cylinder of $f$. Then $f$ factors as
\[X\xto{(-,0)}Z(f)\xto{\ p\ }Y\]
where the first map is a closed embedding and the second the projection. Since the homotopy equivalence $p$ is a weak equivalence and induces isomorphisms on $H_n(-;\Z)$, we can assume without loss of generality that $X$ is a closed subspace of $Y$ (by replacing $Y$ with $Z(f)$).

Since $f_*:\pi_1(X,x)\to\pi_1(Y,x)$ and $f_*:\pi_n(X,x)\to\pi_n(Y,x)$ for $n\ge2$ are isomorphisms, the Hurewicz theorem (in its relative version \ref{theorem:hurewicz}) applies. By the long exact sequence we have that $\pi_k(Y,X,x)=0$ for all $k\ge1$, hence by Hurewicz $H_n(Y,X;\Z)=0$ for all $n\ge1$. So $f_*:H_n(X;\Z)\to H_n(Y;\Z)$ is an isomorphism by the homology long exact sequence.
\end{proof}

Question: what are all natural transformations
\[H^2(-;\Z)\to H^6(-;\Z)\]
as functors $\Top\to\Set$?

Some examples are $x\mapsto x^3= x\smile x\smile x$ (the cup product taken three times) or $x\mapsto kx^3$ for $k\in\Z$. Are these all of them?

\begin{theorem}\label{theorem:cohomology-operations}
Let $A$ be an abelian group, $n\ge0$. Let $F:\Top^\op\to\Set$ be any functor that sends weak equivalences to isomorphisms. Then the map
\begin{align*}
    \Nat_{\Top^\op\to\Set}(H^n(-;A),F)&\to F(K(A,n))\\
    \tau=\cb{\tau_X:H^n(X;A)\to F(X)}&\mapsto \tau_{K(A,n)}(\iota)
\end{align*}
is bijective.
\end{theorem}

\begin{example}
We have a bijection
\[\Nat(H^2(-;\Z),H^6(-;\Z))\xto{\cong}H^6(K(\Z,2);\Z)\cong H^6(\cp{\infty};\Z)\cong\Z\cb{\iota^3}\]
sending $x\mapsto kx^3$ on the left and to $k\iota^3$ on the right.
\end{example}

In general, $\Nat(H^n(-;A),H^m(-;B))\cong H^m(K(A,n);B)$ are called the cohomology operations of type $(A,n,B,m)$.\todo{AT1Sheet10.3 has some interesting stuff about this}

Note:\rightnote{\Attention\ I do not understand well these results, I might have written something dumb here and there.} the cohomology operations of a given type are usually really difficult to compute. A couple of facts about this story: $H^*(K(\FF_2,n);\FF_2)$ for $n\ge2$ is a polynomial algebra in infinitely many explicitly known generators (for $n=1$ on one generator) given by the "Steenrod operations"; for $p$ an odd prime, $H^*(K(\FF_p,n);\FF_p)$ is a polynomial $\otimes$ exterior algebra. We know that $H^m(K(\Z,n);\Z)$ is finitely generated, so it can be decomposed into free and $p$-partition groups; $H^m(K(\Z,n);\Z)\otimes\Z_{(p)}=H^m(K(\Z_{(p)},n),\Z_{(p)})$ related to $A=B=\FF_p$. Also:
\[H^*(K(\Q,n);\Q)\cong\begin{cases}
\Q[\iota] &\text{if } n\text{ is even}\\
\Lambda_\Q[\iota] &\text{if } n\text{ is odd}
\end{cases}
\]

\begin{proof}
The essential ingredients are the Yoneda lemma and CW-approximations, as expected.

Injectivity. Suppose that $\tau$ and $\mu$ are two natural transformations from $H^n(-;A)$ to $F$ such that $\tau_{K(A,n)}(\iota)=\mu_{K(A,n)}(\iota)$.

If $X$ is a CW-complex and $x\in H^n(X;A)$, there is a continuous map $f:X\to K(A,n)$ such that $f^*(\iota)=x$. So by naturality of $\tau$ and $\mu$:
\[\tau_X(x)=\tau_X(f^*(\iota))=f^*(\tau_{K(A,n)}(\iota))=f^*(\mu_{K(A,n)}(\iota))=\mu_X(f^*(\iota))=\mu_X(x),\]
hence $\tau_X=\mu_X$ for all CW-complexes $X$.

If $Y$ is any space, choose a CW-approximation $\alpha:X\xto{\sim} Y$. Let $y\in H^n(Y;A)$. Then by naturality:
\[\alpha^*(\mu_Y(y))=\mu_X(\alpha^*(y))=\tau_X(\alpha^*(y))=\alpha^*(\tau_Y(y))\]
where the middle equality is by the previous special case. Since $\alpha^*:F(Y)\to F(X)$ is an isomorphism by hypothesis, this proves $\mu_Y(y)=\tau_Y(y)$, hence $\mu=\tau$.

Surjectivity. First we need to prove an intermediate result.

Claim. Let $F$ be a (contravariant) functor that sends weak equivalences to isomorphisms. Then $F$ takes homotopic maps to the same map (Warning: the converse is \tit{not} true!).

\begin{claimproof}
Let $f,g:X\to Y$ be two homotopic continuous maps and choose an homotopy $H:X\times[0,1]\to Y$ from $f$ to $g$. Let $i_0,i_1:X\to X\times[0,1]$ be the two \enquote{extremal} inclusions, $p:X\times[0,1]\to X$ the projection. These maps are homotopy equivalences, hence weak equivalences. So $F(p):F(X)\to F(X\times[0,1])$ is an isomorphism. We have
\[F(i_0)\circ F(p)=F(p\circ i_0)=F(\id_X)=F(p\circ i_1)=F(i_1)\circ F(p).\]
Because $F(p)$ is an isomorphism, we conclude that $F(i_0)=F(i_1)$. Now
\[F(f)=F(H\circ i_0)=F(i_0)\circ F(H)=F(i_1)\circ F(H)=F(H\circ i_1)=F(g).\]
\end{claimproof}

To prove surjectivity, for any element $u\in F(K(A,n))$ we will construct a natural transformation $\Phi^u:H^n(-;A)\to F$ such that $\Phi^u_{K(A,n)}(\iota)=u$.

Let $Y$ be a space and choose a CW-approximation $\alpha:X\xto{\sim}Y$. We let $y\in H^n(Y;A)$ be a class. Since $X$ is a CW-complex, there is a continuous map $f:X\to K(A,n)$ such that $\alpha^*(y)=f^*(\iota)$ in $H^n(X;A)$. We define $\Phi^u_Y(y)=(\alpha^*)^\inv(f^*(u))$. Note that we have:
\[F(K(A,n))\xto{f^*}F(X)\underset{\cong}{\xleftarrow{\alpha^*}}F(Y)\]

Claim. $\Phi^u_Y(y)$ is independent of the choices of $\alpha$ and $f$.

\begin{claimproof}
Let $\beta:X'\xto{\sim}Y$ be another CW-approximation. Let $g:X'\to K(A,n)$ be another continuous map such that $\beta^*(y)=(f')^*(\iota)$.

By uniqueness of CW-approximations there is a homotopy equivalence $\psi:X\to X'$ such that
\[
\begin{tikzcd}
X \ar[dr,"\alpha"'] \ar[rr,"\psi"] & & X' \ar[dl,"\beta"]\\
& Y &
\end{tikzcd}
\]
commutes up to homotopy. Then:
\[(g\circ\psi)^*(\iota)=\psi^*((g)^*(\iota))=\psi^*(\beta^*(y))=(\beta\circ\psi)^*(y)=\alpha^*(y)=f^*(\iota).\]

Because $X$ is a CW-complex, the maps $g\circ\psi,f:X\to K(A,n)$ are homotopic. Hence we have $F(f)=F(g\circ\psi)=F(\psi)\circ F(g)$. So the following diagram commutes:
\[
\begin{tikzcd}[row sep=small,column sep={10em,between origins}]
& F(X) & \\
F(K(A,n)) \ar[ur,"f^*"] \ar[dr,"g^*"'] & & F(Y) \ar[ul,"\alpha^*"',"\cong"] \ar[dl,"\beta^*","\cong"']\\
& F(X') \ar[uu,"\cong","\psi^*"']
\end{tikzcd}
\]
So the potentially different definitions of $\Phi^u_Y(y)$ agree.
\end{claimproof}

Clearly $\Phi^u_{K(A,n)}(\iota)=u$ since we can take both $\alpha$ and $f$ to be $\id:K(A,n)\to K(A,n)$. Hence we are left to prove that $\Phi^u$ is indeed a natural transformation.

Claim. $\Phi^u=\cb{\Phi^u_Y}$ is a natural transformation.

\begin{claimproof}
Let $h:Y\to Z$ be any continuous map, and let $z\in H^n(Z;A)$. Let $\alpha:X\to Y$ be any CW-approximation to $Y$. We choose a CW-approximation for $Z$ relative to the map $h\circ\alpha:X\to Z$ (which we can do by using theorem \ref{theorem:cw-approximation} in the general, i.e. relative to a map, version). This is a CW-complex $\bar X$ containing $X$ as a subcomplex and a weak equivalence $\beta:\bar X\to Z$ such that $\beta|_X=h\circ\alpha$. Consider the following diagram:
\[
\begin{tikzcd}
& X \ar[dl,"g"] \ar[d,hook,"i"]\ar[r,"\sim"',"\alpha"] & Y \ar[d,"h"]\\
K(A,n) & \bar X \ar[l,"f"] \ar[r,"\sim","\beta"'] & Z
\end{tikzcd}
\]
where $f:\bar X\to K(A,n)$ is any continuous map such that $f^*(\iota)=\beta^*(z)$ and $g=f|_X$. The map $g:X\to K(A,n)$ satisfies:
\[g^*(\iota)=(f\circ i)^*(\iota)=i^*(f^*(\iota))=i^*(\beta^*(z))=\alpha^*(h^*(z)),\]
hence we can use $\alpha:X\to Y$, $g:X\to K(A,n)$ to define $\Phi^u_Y(h^*(z))$. Then we have: \[\Phi^u_Y(h^*(z))=(\alpha^*)^\inv(i^*(f^*(u)))=h^*((\beta^*)^\inv(f^*(u)))=h^*(\Phi^u_Z(z)),\]
i.e. $\Phi^u$ is a natural transformation.
\end{claimproof}
\end{proof}

What are all natural transformations, on paracompact spaces, of functors $\Top_\text{para}\to\Set$, $\Pic_\R\to\Pic_\CC$?

Surely we have:

\[[\xi:L\to X]\mapsto[\xi_\CC:L\otimes_\R\CC\to X],\]
\[[\xi:L\to X]\mapsto[X\times\CC\xto{\pr} X].\]
These two turn out to be all natural operations:
\[\Nat(\Pic_\R,\Pic_\CC)\cong\Nat(H^1(-,\FF_2),\Pic_\CC)\cong\Pic_\CC(\rp{\infty})\cong H^2(\rp{\infty};\Z)\cong\Z/2.\]

\input{Chapters/AT1-Lecture19}
\input{Chapters/AT1-Lecture20}
\input{Chapters/AT1-Lecture21}
\input{Chapters/AT1-Lecture22}
% Lecture 23

\section{Equivalences of Homotopy Categories}

\lecture[We finally get to the result we wanted. Spaces and simplicial sets are essentially (at least for the purposes of homotopy theory) the same!]{17-01-2022}

So far we have proved that the realization of every simplicial set admits a CW-structure (theorem \ref{theorem:cw-structure-on-realization}), that the singular complex functor is homotopical, i.e. it takes weak equivalences to homotopy equivalences (theorem \ref{theore:singular-complex-functor-is-homotopical}), and that the adjunction counit $\epsilon_Z:|\S(Z)|\to Z$ is a weak homotopy equivalence (theorem \ref{theorem:adjunction-counit-is-a-weak-equivalence}).

We can finally prove that the geometric realization and singular complex functors descend to equivalences of homotopy categories of topological spaces and simplicial sets. More precisely, we will see that there is a diagram of equivalence of categories:

\[
\begin{tikzcd}
    \Ho(\Top_\text{CW}) \ar[dd,bend right,"\text{inclusion}"',"\cong"] & \\
     & \sSet[\weq^\inv] \ar[ul,bend right,"|-|"',"\cong"]\\
     \Top[\weq^\inv] \ar[ur,bend right, "\S"',"\cong"] & 
\end{tikzcd}
\]

In the diagram, $\ho(\Top_\text{CW})$ is the \tbf{homotopy category of CW-complexes.} Its objects are all spaces that admit the structure of a CW-complex; its morphisms are homotopy classes of continuous maps. The two other categories are certain localizations, a notion that we will now introduce.

\subsection{Localization of Categories}

Let $\Cc$ be a category and $\Ww$ a class of morphisms of $\Cc$. A functor $F:\Cc\to\Dd$ is $\Ww$\tbf{-inverting} if it sends all morphisms in $\Ww$ to isomorphisms in $\Dd$. A \tbf{localization of }$\Cc$\tbf{ at }$\Ww$ is a functor $\gamma:\Cc\to\Cc[\Ww^\inv]$ that is initial among $\Ww$-inverting functors.

In more detail, $\gamma:\Cc\to\Cc[\Ww^\inv]$ is a localization of $\Cc$ at $\Ww$ if and only if:
\begin{itemize}[label={-}]
    \item $\gamma$ is $\Ww$-inverting,
    \item $\gamma$ has the property that for every $\Ww$-inverting functor $F:\Cc\to\Dd$, there exists a unique functor $G:\Cc[\Ww^\inv]\to\Dd$ such that $G\circ\gamma=F$.
\end{itemize}
\[
\begin{tikzcd}
\Cc \ar[dr,"F"] \ar[rr,"\gamma"] & & {\Cc[\Ww^\inv]} \ar[dl,dashed,"\exists! G"]\\
& \Dd &
\end{tikzcd}
\]

Before going on, there is a number of important observations to make on localizations.

\begin{itemize}
    \item Localizations, if they exist, are unique up to preferred isomorphism (as is always the case with universal constructions). The usual drill. Suppose $\gamma:\Cc\to\Dd$ and $\mu:\Cc\to\Ee$ are two localizations at the same class of morphisms $\Ww$. Then $\gamma$ and $\mu$ are $\Ww$-inverting, so the universal properties provide unique functors $G:\Dd\to\Ee$ and $H:\Ee\to\Dd$ such that $G\circ\gamma=\mu$ and $H\circ\mu=\gamma$.
    \[
    \begin{tikzcd}
     & \Cc \ar[dr,"\mu"] \ar[dl,"\gamma"] \ar[rr,"\gamma"] & & \Dd\\
     \Dd \ar[rr,dashed,"\exists!G"] & & \Ee \ar[ur,dashed,"\exists!H"] &
    \end{tikzcd}
    \]
    Then $H\circ G$ satisfies
    \[H\circ G\circ\gamma=H\circ\mu=\gamma=\id_\Dd\circ\gamma,\]
    hence $H\circ G=\id_\Dd$ by uniqueness. Similarly $G\circ H=\id_\Ee$.
    
    \item A brief warning about set-theoretic issues: in principle there may be some when working with localizations (the \href{https://www.math.uni-bonn.de/people/schwede/sset_vs_spaces.pdf}{official notes} give one, i.e. localizations always exist but up to some fiddling with universes), but in our case we are going to give concrete constructions for everything, hence we will stick to a naive approach.
    
    \item Localizations are bijective on objects. Let $X$ be a set and let $EX$ be the category with object set $X$ and a unique morphism $(y,x):x\to y$ for each pair of objects. Then the morphism $(x,y):y\to x$ is inverse to $(y,x)$, so $EX$ is a groupoid. $EX$ is called the \tbf{indiscrete category} with object set $X$.
    
    Since\rightnote{Last remark on set-theoretic issues: I feel a little bit uneasy with the argument we use to show that localization is bijective on objects, but I will set aside my qualms about foundations until I have the time to read something about it.} $EX$ is a groupoid, every functor $\Cc\to EX$ is $\Ww$-inverting, hence it extends uniquely over $\gamma$ to a functor $\Cc[\Ww^\inv]\to EX$. However, we also have that any map $\ob(\Cc)\to X=\ob(EX)$ can be uniquely extended to a functor $\Cc\to EX$. Altogether we have that any map $\ob(\Cc)\to X$ extends uniquely over $\gamma$ to a map $\ob(\Cc[\Ww^\inv])\to X$, which means that $\ob(\gamma):\ob(\Cc)\to\ob(\Cc[\Ww^\inv])$ is bijective.
    
    \item As a consequence of the previous item, one can always chose a localization of $\Cc$ at $\Ww$, if it exists, to satisfy $\ob(\Cc[\Ww^\inv])=\ob(\Cc)$ with $\gamma$ the identity on objects.
    
    \item Localizations of categories are somewhat analogous to localizations of rings: a localization of a ring $R$ at a subset $S$ is a ring homomorphism $\gamma:R\to R[S^\inv]$ that is initial among $S$-inverting ring morphisms.
    
    If the ring $R$ is commutative and $S$ is closed under multiplication, then a localization can be constructed by means of \enquote{fractions}: $R[S^\inv]$ are equivalence classes of pairs $(r,s)\in R\times S$, where the equivalence is that $(r,s)\sim(r',s')$ when there is a $t\in S$ such that $rs't=r'st$; then the equivalence class of $(r,s)$ is written $r/s$ and
    \[R\to R[S^\inv],\ r\mapsto \frac{r}{1}\]
    is a localization.
    
    \item In the context of non-commutative rings, localizations exist but there need not be any concrete description of $R[S^\inv]$ that resembles fractions. There exist sufficient conditions (known as \enquote{calculus of fractions} or the \enquote{\href{https://en.wikipedia.org/wiki/Ore_condition}{Ore condition}}) that give a \enquote{fraction-like} description of $R[S^\inv]$. For the calculus of fractions, a reference given by the Professor is a book by Gabriel and Zisman \cite{gabriel-zisman}.
\end{itemize}

A localization $\gamma:\Cc\to\Cc[\Ww^\inv]$ induces a bijection from the set of functors $\Cc[\Ww^\inv]\to \Dd$ to the set of $\Ww$-inverting functors $\Cc\to\Dd$. The next proposition shows that this is also true of natural transformations.

Note: for categories $\Cc,\Dd$, we write $\Fun(\Cc,\Dd)$ for the category of functors $\Cc\to\Dd$ and natural transformations between them.

\begin{proposition}\label{proposition:localization-induces-embedding}
Let $\gamma:\Cc\to\Cc[\Ww^\inv]$ be a localization at a class of morphisms $\Ww$. Then for every category $\Dd$ the functor
\[\Fun(\gamma,\Dd):\Fun(\Cc[\Ww^\inv],\Dd)\to\Fun(\Cc,\Dd)\]
is an isomorphism onto the full subcategory of $\Fun(\Cc,\Dd)$ spanned by the $\Ww$-inverting functors.
\end{proposition}

\begin{proof}
On objects, this is the defining universal property. At the level of morphisms, we exploit the fact that natural transformations can be interpreted as functors, as follows. Let $I$ be the category with two objects $0$ and $1$ and only one non-identity morphism $a:0\to1$. Functors $\Cc\to\Fun(I,\Dd)$ correspond to natural transformations of functors $\Cc\to\Dd$.
Indeed, suppose $\tau:F\to G$ is a natural transformation of functors $F,G:\Cc\to\Dd$. This yields a single functor $\tau^\flat:\Cc\to\Fun(I,\Dd)$ by
\begin{itemize}[label={-}]
    \item on objects:
    \[\tau^\flat(c)(0)=F(c),\ \tau^\flat(c)(1)=G(c),\]
    \[\tau^\flat(c)(0\to1)=(\tau_c:F(c)\to G(c)),\]
    \item on morphisms:
    \[\tau^\flat(f:c\to c')(0)=F(f),\]
    \[\tau^\flat(f:c\to c')(1)=G(f).\]
\end{itemize}

Now we can apply the localization property of $\gamma:\Cc\to\Cc[\Ww^\inv]$ to functors with target $\Fun(I,\Dd)$. We get a bijection between the set of functors $\Cc[\Ww^\inv]\to\Fun(I,\Dd)$ and the set of $\Ww$-inverting functors $\Cc\to\Fun(I,\Dd)$. Translating functors to $\Fun(I,\Dd)$ to natural transformations of functors $\Cc\to\Dd$, this becomes the statement that precomposition with $\gamma$ is a bijection from the set of natural transformations of functors $\Cc[\Ww^\inv]\to\Dd$ to the set of natural transformations with $\Ww$-inverting functors $\Cc\to\Dd$. But this is exactly the statement that
\[\Fun(\gamma,\Dd):\Fun(\Cc[\Ww^\inv],\Dd)\to\Fun(\Cc,\Dd)\]
is bijective on $\Hom$-sets.
\end{proof}

\begin{remark}
There is a weaker notion of localization that takes seriously the fact that categories form a $2$-category.

A functor $\gamma:\Cc\to\Cc[\Ww^\inv]$ is a \tbf{weak localization} at $\Ww$ if for every category $\Dd$, the functor
\[\Fun(\gamma,\Dd):\Fun(\Cc[\Ww^\inv],\Dd)\to\Fun(\Cc,\Dd)\]
is an equivalence onto the full subcategory of $\Ww$-inverting functors.

Proposition \ref{proposition:localization-induces-embedding} shows that localizations are weak localizations, but the latter are more general: if $\gamma:\Cc\to\Cc[\Ww^\inv]$ is a localization and $F:\Cc[\Ww^\inv]\to\Ee$ an equivalence of categories that is not an isomorphism, then the composite $F\gamma:\Cc\to\Ee$ is a weak localization, but not a localization, at $\Ww$.
\end{remark}

\subsection{Construction of a Localization of Top}

We define a category $\Top[\weq^\inv]$ as follows:
\begin{itemize}
    \item the objects are all topological spaces,
    \item the morphisms are $\Hom_\Top(|\S(A)|,|\S(B)|)$ modulo homotopies of maps,
    \item composition is composition of homotopy classes of maps.
\end{itemize}

We define a functor $\gamma:\Top\to\Top[\weq^\inv]$ as follows:
\begin{itemize}[label={-}]
    \item on objects $\gamma(A)=A$,
    \item on morphisms $\gamma(f:A\to B)=[\,|\S(f)|:|\S(A)|\to|\S(B)|\,]$.
\end{itemize}

The functor $\gamma:\Top\to\Top[\weq^\inv]$ inverts weak equivalences. Indeed, let $f:A\to B$ be a weak equivalence, then $\S(f):\S(A)\to\S(B)$ is a homotopy equivalence of simplicial sets, so $|\S(f)|$ is a homotopy equivalence by Whitehead and this is an isomorphism in $\Top[\weq^\inv]$.

\begin{theorem}\label{theorem:localization-of-top}
The functor $\gamma:\Top\to\Top[\weq^\inv]$ is a localization at the class of weak homotopy equivalences.
\end{theorem}

\begin{warning}
The proof we saw in the lecture had a problem: the diagram $(*)$ was assumed to be commutative (it deceptively looks like a naturality square), while it only commutes up to homotopy. I am hence following the proof in the \href{https://www.math.uni-bonn.de/people/schwede/sset_vs_spaces.pdf}{official notes}, but this requires referencing a result that we saw in the \tit{following} lecture (proposition \ref{proposition:adjunction-unit-is-a-weak-equivalence}) and one that we did not see at all (proposition \ref{proposition:calculus-of-fractions-in-sset}, which I have added where it seemed to fit best).
\end{warning}

\begin{proof}
We let $F:\Top\to\Dd$ be any functor that inverts weak equivalences. We must show that there is a unique functor $G:\Top[\weq^\inv]\to\Dd$ such that $G\circ\gamma=F$.

First we note that morphisms in $\Top[\weq^\inv]$ can be written as \enquote{fractions} as follows. For a continuous map $\alpha:|\S(A)|\to|\S(B)|$ we get a square in $\Top$:
\[
\begin{tikzcd}[column sep=large]
 {|\S|\S(A)||} \ar[d,"{|\S(\alpha)|}"'] \ar[r,"{|\S(\epsilon_A)|}"] & {|\S(A)|} \ar[d,"\alpha"]\\
 {|\S|\S(B)||} \ar[r,"{|\S(\epsilon_B)|}"] & {|\S(B)|}
\end{tikzcd}\tag{$*$}
\]
This diagram will typically \tit{not} commute (it is not a naturality square). We argue, however, that the diagram commutes up to homotopy. To this end we consider the square:
\[
\begin{tikzcd}[column sep=large]
 {|\S(A)|} \ar[d,"\alpha"] \ar[r,"{|\eta_{\S(A)}|}"] & {|\S|\S(A)||} \ar[d,"{|\S(\alpha)|}"] \ar[r,"{|\S(\epsilon_A)|}"] & {|\S(A)|} \ar[d,"\alpha"]\\
 {|\S(B)|} \ar[r,"{|\eta_{\S(B)}|}"] & {|\S|\S(B)||} \ar[r,"{|\S(\epsilon_B)|}"] & {|\S(B)|}
\end{tikzcd}
\]
Then we get
\begin{multline*}
    |\S(\epsilon_B)|\circ|\S(\alpha)|\circ|\eta_{\S(A)}|\sim|\S(\epsilon_B)|\circ|\eta_{\S(B)}|\circ\alpha=|\S(\epsilon_B)\circ\eta_{\S(B)}|\circ\alpha=\\
    =\alpha\circ|\S(\epsilon_A)\circ\eta_{\S(A)}|=\alpha\circ|\S(\epsilon_A)|\circ|\eta_{\S(A)}|
\end{multline*}
where the first homotopy is provided by proposition \ref{proposition:calculus-of-fractions-in-sset} for $X=\S(A)$ and $Y=\S(B)$ and we use two instances of the \href{http://nlab-pages.s3.us-east-2.amazonaws.com/nlab/show/triangle+identities}{triangle identities} of the adjunction $|\!-\!|\dashv\S$. The morphism $\eta_{\S(A)}$ is a weak equivalence by proposition \ref{proposition:adjunction-unit-is-a-weak-equivalence}, so $|\S(A)|$ is a homotopy equivalence. We can thus \enquote{cancel} $|\eta_{\S(A)}|$ up to homotopy, and the previous equality implies that the maps
\[|\S(\epsilon_B)|\circ|\S(\alpha)|,\alpha\circ|\S(\epsilon_A)|:|\S|\S(A)|\to|\S(B)|\]
are homotopic.

Since $(*)$ commutes up to homotopy, we can pass to homotopy classes to obtain a commutative square (in $\Top[\weq^\inv]$):
\[
\begin{tikzcd}
 {|\S(A)|} \ar[d,"\gamma(\alpha)"'] \ar[r,"\gamma(\epsilon_A)"] & A \ar[d,"{[\alpha]}"]\\
 {|\S(B)|} \ar[r,"\gamma(\epsilon_B)"] & B
\end{tikzcd}
\]
Since $\epsilon_X$ and $\epsilon_Y$ are weak equivalences, $\gamma$ inverts them, hence
\[[\alpha]=\gamma(\epsilon_B\circ\alpha)\circ\gamma(\epsilon_A)^\inv.\]

Now we return to our original problem: we have a functor $F:\Top\to\Dd$ which inverts weak equivalences and we want an unique $G:\Top[\weq^\inv]\to\Dd$ such that $G\circ\gamma=F$.

Uniqueness. On objects $G$ is uniquely determined because we must have
\[F(A)=G(\gamma(A))=G(A).\]
To see that $G$ is also uniquely determined on morphisms, let $[\alpha]:A\to B$ be a morphism in $\Top[\weq^\inv]$, with $\alpha:|\S(A)|\to|\S(B)|$ a continuous map. Then
\[G[\alpha]=G(\gamma(\epsilon_B\circ\alpha)\circ\gamma(\epsilon_A)^\inv)=G(\gamma(\epsilon_B\circ\alpha))\circ G(\gamma(\epsilon_A))^\inv=F(\epsilon\circ\alpha)\circ F(\epsilon_A)^\inv\]
so $G$ is determined by $F$ also on morphisms.

Existence. With the uniqueness part of the proof in mind, we construct $G$ as follows. On objects,
\[G(A):=F(A).\]
On morphisms (given that $F$ inverts weak equivalences),
\[G\,[\,\alpha:|\S(A)|\to|\S(B)|{\,}]:=F(\epsilon_B)\circ F(\alpha)\circ F(\epsilon_A)^\inv.\]
To show that $G$ is well-defined we need to prove that $F:\Top\to\Dd$ takes the same value on homotopic maps. We use an argument which we have already seen (essentially identical) in the proof of proposition \ref{theorem:cohomology-operations}. Let $H:A\times[0,1]\to B$ be an homotopy from $f=H(-,0)$ to $g=H(-,1)$. We let $i_0,i_1:A\to A\times[0,1]$ be the \enquote{end inclusions}, $p:A\times[0,1]\to A$ the projection. Then $p$ is a weak equivalence, so $F(p)$ is an isomorphism. We have
\[F(p)\circ F(i_0)=F(p\circ i_0)=F(\id_A)=F(p\circ i_1)=F(p)\circ F(i_1)\]
and $F(p)$ is an isomorphism, hence $F(i_0)=F(i_1)$. Then
\[F(f)=F(H\circ i_0)=F(H)\circ F(i_0)=F(H)\circ F(i_1)=F(H\circ i_1)=F(g).\]

Now we need to check that $G$ is actually functorial. Let $\beta:|\S(B)|\to|\S(D)|$ be a continuous map. We have
\begin{align*}
    G[\beta]\circ G[\alpha]&=F(\epsilon_D)\circ F(\beta)\circ F(\epsilon_B)^\inv\circ F(\epsilon_B)\circ F(\alpha)\circ F(\epsilon_A)^\inv\\
    &=F(\epsilon_D)\circ F(\beta\circ\alpha)\circ F(\epsilon_A)^\inv\\
    &=G[\beta\circ\alpha]=G([\beta]\circ[\alpha]),
\end{align*}
and clearly $G[\id_{|\S(A)|}]=\id_{G(A)}$.

Finally, it remains to show that $G\circ\gamma=F$. This is clear on objects. To see it on morphisms, let $f:X\to Y$ be any continuous map, then
\[G(\gamma(f))=G[\,|\S(f)|{\,}]=F(\epsilon_B)\circ F(|\S(f)|)\circ F(\epsilon_A)^\inv=F(f)\circ F(\epsilon_A)\circ F(\epsilon_A)^\inv=F(f),\]
using naturality of the adjunction counit.
\end{proof}

% Lecture 24

\lecture[This lecture is very similar to the previous one, with simplicial sets in place of spaces.]{2022-01-19}

We can now show that there is an equivalence $\Top[\weq^\inv]$ and $\Ho(\Top_\text{CW})$.

The functor $|-|\circ\S:\Top\to\Top$ takes weak equivalences to homotopy equivalences and takes values in $\Top_\text{CW}$, so the composite
\[\Top\xto{|-|\circ\S}\Top_\text{CW}\xto{\text{proj}}\Ho(\Top_\text{CW})\]
inverts weak equivalences. The universal property of the localization then provides a unique functor
\[\Phi:\Top[\weq^\inv]\to\Ho(\Top_\text{CW})\]
such that the following square of categories and functors commutes
\[
\begin{tikzcd}
\Top \ar[d,"\gamma"] \ar[r,"\S"] & \sSet \ar[r,"{|-|}"] & \Top_\text{CW} \ar[d,"\text{proj}"]\\
\Top[\weq^\inv] \ar[rr,dashed,"\exists!\Phi"] & & \Ho(\Top_\text{CW})
\end{tikzcd}
\]
and we claim that this is an equivalence of categories.

\begin{theorem}
The functor $\Phi$ is an equivalence of categories.
\end{theorem}

\begin{proof}
Unraveling the definitions, we see that on objects $\Phi$ is given by $\Phi(A)=|\S(A)|$ and on morphisms $\Phi$ is
\[\Top[\weq^\inv](A,B)=\Ho(\Top_\text{CW})(|\S(A)|,|\S(B)|)\xto{\id}\Ho(\Top_\text{CW}(|\S(A)|,|\S(B)|).\]
So $\Phi$ is fully faithful. We want to show that $\Phi$ is essentially surjective on objects. Let $K\in\Ho(\Top_\text{CW})$ be any object. Then the adjunction counit $\epsilon_K:|\S(K)|\to K$ is a weak equivalence, hence an homotopy equivalence by the Whitehead theorem, since source and target admit CW-structures. So the homotopy class of $\epsilon_K$ is an isomorphism in $\Ho(\Top_\text{CW})$
\[[\epsilon_K]:|\S(K)|=\Phi(K)\xto{\cong}K,\]
i.e. $\Phi$ is (fully faithful and) essentially surjective, hence an equivalence of categories.
\end{proof}

\subsection{Construction of a Localization of sSet}

We define weak equivalences in $\sSet$ in the following way: a morphism  of simplicial sets $f:X\to Y$ is a \tbf{(simplicial) weak equivalence} if $|f|:|X|\to|Y|$ is a homotopy equivalence.

\begin{examples}
Since geometric realization preserves homotopy relations (see AT1Sheet11.2 or the proof of proposition \ref{proposition:adjunction-preserves-homotopy}, where it is implicit), homotopy equivalences of simplicial sets are weak equivalences. But weak equivalences are typically not homotopy equivalences.
Let $f:\de\Delta^2\to\Delta^1/\de\Delta^1$
be the morphism such that $d_0,d_1\in(\de\Delta^1)_1$ map to the degenerate $1$-simplex of $\Delta^1/\de\Delta^1$ and $d_2$ maps to the non-degenerate $1$-simplex of $\Delta^1/\de\Delta^1$.\alvaropls

Then $f$ is a weak equivalence, because under the homeomorphisms
\[|\de\Delta^2|\cong\de\sx{1}\ \text{ and }\ |\Delta^1/\de\Delta^1|\cong\sx{1}/\de\sx{1}\cong[0,1]/\cb{0,1}\]
the realization of $f$ is the continuous map $\de\sx{2}\to[0,1]/\cb{0,1}$ that identifies two of the three sides of $\de\sx{2}$ to a point and maps the third side linearly onto the target, which is a homotopy equivalence. But $f$ is not a simplicial homotopy equivalence. Indeed, suppose we have a morphism of simplicial sets $g:\Delta^1/\de\Delta^1\to\de\Delta^2$, then $g$ must send the non-degenerate $1$-simplex $\id_{[1]}\in\Delta^1/\de\Delta^1$ to a $1$-simplex of $\de\Delta^2$ whose two vertices are the same: only the degenerate simplices have this property. Then $g$ must be constant at one of the three vertices of $\de\Delta^2$ and so $|g|$ a constant map at one of the $3$ vertices of $\de\sx{2}$. In particular, $|g|$ is not a homotopy equivalence.

More generally, examples of weak equivalences that are not homotopy equivalences are the maps
\[\de\Delta^n\to\Delta^{n-1}/\de\Delta^{n-1}\]
that collapse one of the horns.
\end{examples}

\begin{proposition}\label{proposition:adjunction-unit-is-a-weak-equivalence}
For every simplicial set $X$, the adjunction unit $\eta_X:X\to\S|X|$ is a weak equivalence.
\end{proposition}

\begin{proof}
One of the triangle identities of an adjunction show that the composite
\[|X|\xto{|\eta_X|}|\S|X||\xto{\epsilon_{|X|}}|X|\]
is the identity. Then $|\eta_X|$ is a homotopy equivalence, hence a weak homotopy equivalence.\rightnote{We are using that the counit is a weak equivalence and Whitehead.}
\end{proof}

As we already did for $\Top$, we want to construct a localization of $\sSet$ at the class of weak equivalences.

We define the category $\sSet[\weq^\inv]$ as follows:
\begin{itemize}
    \item the objects are all simplicial sets,
    \item the morphisms are $\Hom_{\Ho(\Top_\text{CW})}(|X|,|Y|)$,
    \item composition is composition of homotopy classes of maps.
\end{itemize}

We define a functor $\gamma:\Top\to\Top[\weq^\inv]$ as follows:
\begin{itemize}[label={-}]
    \item on objects $\gamma(X)=X$,
    \item on morphisms $\gamma(f:X\to Y)=[\,|f|:|X|\to |Y|\,]$.
\end{itemize}

\begin{warning}
Here I add a proposition which is in the \href{https://www.math.uni-bonn.de/people/schwede/sset_vs_spaces.pdf}{official notes}, but we did not see in class.
\end{warning}

The next proposition shows that it is possible to write any morphism in $\sSet[\weq^\inv]$ as a \enquote{fraction} of a morphism of simplicial sets and the inverse of a weak equivalence. Note that we already showed the same for $\Top[\weq^\inv]$ in the proof of theorem \ref{theorem:localization-of-top}, but in fact in that proof we appeal to the results we are about to show. Observe also that this is a special property of the localizations we are working with, but it does not hold in general.

\begin{proposition}\label{proposition:calculus-of-fractions-in-sset}
Let $X$ and $Y$ be simplicial sets, $\alpha:|X|\to|Y|$ a continuous map between their geometric realizations.
\begin{numerate}
    \item The square of spaces and continuous maps
    \[
    \begin{tikzcd}
    {|X|} \ar[d,"\alpha"] \ar[r,"{|\eta_X|}"] & {|\S|X||} \ar[d,"{|\S(\alpha)|}"]\\
    {|Y|} \ar[r,"{|\eta_Y|}"] & {|\S|Y||}
    \end{tikzcd}\tag{$*$}
    \]
    commutes up to homotopy.
    
    \item The relation
    \[[\alpha]=\gamma(\eta_Y)^\inv\circ\gamma(\S(\alpha)\circ\eta_X)\]
    holds as morphisms from $X$ to $Y$ in $\sSet[\weq^\inv]$.
\end{numerate}
\end{proposition}

\begin{proof}
$(1)$ Composing with the adjunction counit, from $(*)$ we get the diagram:
\[
    \begin{tikzcd}
    {|X|} \ar[d,"\alpha"] \ar[r,"{|\eta_X|}"] & {|\S|X||} \ar[d,"{|\S(\alpha)|}"] \ar[r,"\epsilon_{|X|}"] & {|X|} \ar[d,"\alpha"]\\
    {|Y|} \ar[r,"{|\eta_Y|}"] & {|\S|Y||} \ar[r,"\epsilon_{|Y|}"] & {|Y|}
    \end{tikzcd}
\]
where the right half commutes by naturality. Then, thanks to the triangle identities of the adjunction $|\!-\!|\dashv\S$, we have
\[\epsilon_{|Y|}\circ|\S(\alpha)|\circ|\eta_X|=\alpha\circ\epsilon_{|X|}\circ|\eta_X|=\epsilon_{|Y|}\circ|\eta_Y|\circ\alpha\]
The map $\epsilon_{|Y|}$ is a weak equivalence between CW-complexes, hence a homotopy equivalence. We can thus \enquote{cancel} $\epsilon_|Y|$ up to homotopy, and the previous equality implies that the maps
\[|\S(\alpha)|\circ|\eta_X|,|\eta_Y|\circ\alpha:|X|\to|\S|Y||\]
are homotopic.

$(2)$ Since $(*)$ commutes up to homotopy, the associated square of homotopy classes commutes, hence we get a commutative square in $\sSet[\weq^\inv]$:
\[
\begin{tikzcd}
X \ar[d,"{[\alpha]}"'] \ar[r,"\gamma(\eta_X)"] & {\S|X|} \ar[d,"\gamma(\S(\alpha))"]\\
Y \ar[r,"\gamma(\eta_Y)"] & {\S|X|}
\end{tikzcd}
\]
Equivalently, we have $\gamma(\eta_Y)\circ[\alpha]=\gamma(\S(\alpha)\circ\eta_X)$. The morphism $\eta_Y:Y\to\S|Y|$ is a weak equivalence by proposition \ref{proposition:adjunction-unit-is-a-weak-equivalence}, so $\gamma(\eta_Y)$ is an isomorphism, and the desired relation follows.
\end{proof}

\begin{theorem}\label{theorem:localization-of-sset}
The functor $\gamma:\sSet\to\sSet[\weq^\inv]$ is a localization at the class of weak equivalences.
\end{theorem}

\begin{proof}
We already know that, essentially by construction, $\gamma$ inverts weak equivalences. The rest of the proof (very) closely resembles the proof we have seen for the localization of $\Top$ (theorem \ref{theorem:localization-of-top}). We just give an outline.

Let $F:\sSet\to\Dd$ be any functor that inverts weak equivalences, we must show that there is an unique functor $G:\sSet[\weq^\inv]\to\Dd$ such that $F=G\gamma:\sSet\to\Dd$.

Uniqueness. The functor $G$ is clearly uniquely determined by $F$ on objects and the \enquote{fraction relation} of proposition \ref{proposition:calculus-of-fractions-in-sset} shows that it is uniquely determined by $F$ also on morphisms.

Existence. The uniqueness argument tells us how to define the functor $G$: we must set $G(X):=F(X)$ on objects and $G[\alpha]:=F(\eta_Y)^\inv\circ F(\S(\alpha))\circ F(\eta_X)$ on morphisms (where we observe that $F$ takes $\eta_Y$ to an isomorphism, since by proposition \ref{proposition:adjunction-unit-is-a-weak-equivalence} the adjunction unit $\eta_Y$ is a weak equivalence).

To show well-definedness, the argument that we used in the proof of theorem \ref{theorem:localization-of-top} (and also of theorem \ref{theorem:cohomology-operations}) can be easily adapted to the context of simplicial sets.

Similarly, functoriality of $G$ and the relation $G\gamma=F:\sSet\to\Dd$ can be shown as in the proof of theorem \ref{theorem:localization-of-top}.
\end{proof}

\subsection{The Promised Equivalence}

We are now ready to construct the equivalence between $\Top[\weq^\inv]$ and $\sSet[\weq^\inv]$.

The geometric realization functor and the singular complex functor both preserve weak equivalences, hence they uniquely descend to the localizations as follows
\[
\begin{tikzcd}[column sep=large]
\sSet \ar[d,"\gamma"] \ar[r,"{|-|}"] & \Top \ar[d,"\gamma"] \ar[r,"\S"] & \sSet \ar[d,"\gamma"]\\
\sSet[\weq^\inv] \ar[r,dashed,"\exists!\alpha"] & \Top[\weq^\inv] \ar[r,dashed,"\exists!\beta"] & \sSet[\weq^\inv]
\end{tikzcd}
\]

\begin{theorem}
The two composite functors $\beta\circ\alpha:\sSet[\weq^\inv]\to\sSet[\weq^\inv]$ and $\alpha\circ\beta:\Top[\weq^\inv]\to\Top[\weq^\inv]$ are naturally isomorphic to their respective identity functors. In particular, they are equivalence of categories.
\end{theorem}

\begin{proof}
We give the argument only for $\beta\circ\alpha$, the other being completely analogous. Composing the adjunction unit $\eta:\id_\sSet\to\S\circ|-|$ with the functor $\gamma:\sSet\to\sSet[\weq^\inv]$ yields a natural transformation
\[\gamma\circ\eta:\gamma\to\gamma\circ\S\circ|-|=\beta\circ\gamma\circ|-|=\beta\circ\alpha\circ\gamma\]
between two functors $\sSet\to\sSet[\weq^\inv]$ that both invert weak equivalences. The universal property of $\gamma$ for natural transformations (proposition \ref{proposition:localization-induces-embedding}) provides a unique natural transformation \[\tau:\id_{\sSet[\weq^\inv]}\to\beta\circ\alpha\]
such that $\gamma\circ\eta=\tau\circ\gamma$. For every simplicial set $X$, this means that
\[\tau_X=\tau_{\gamma(X)}=\gamma(\eta_X)\]
is an isomorphism, since the adjunction unit is a weak equivalence. So $\tau$ is a natural isomorphism.
\end{proof}

\comment{
Note: the previous argument applies very generally

\begin{theorem}
The induced functors $\alpha:\Cc[\Ww^\inv]\to\Dd[\Vv^\inv]$ and $\beta:\Dd[\Vv^\inv]\to\Cc[\Ww^\inv]$ are equivalences.
\end{theorem}
}

\begin{corollary}
There is an equivalence of categories
\begin{align*}
    \sSet[\weq^\inv]&\cong\Ho(\Top_\text{\upshape CW})
\end{align*}
given by $X\mapsto|X|$ on objects and $[\alpha:|X|\to|Y|]\mapsto[\alpha]$ on morphisms.
\end{corollary}

\begin{remark}
There is another category that is equivalent to the previous ones:
\[\Ho(\sSet_\text{Kan})\]
the homotopy category of \tbf{Kan-complexes}. But this is another story...
\end{remark}

% Lecture 25

\chapter{A Taste of Infinity Categories}

\lecture[Cool stuff.]{2022-01-24}

Reference: \href{https://arxiv.org/abs/1007.2925}{A Short Course on $\infty$-Categories} by Moritz Groth.

\warning\ \textcolor{red}{This is just a stub! In particular, while the material on simplicial sets in these notes might be good for beginners, I strongly advise against reading this chapter: there are many details I never got around filling (and probably never will).}

We will talk about quasi-categories, which are a model for \enquote{$\infty$-categories}/\enquote{$(\infty,1)$-categories}.

This theory began with Boardman and Vogt in 1973, but then it remained dormant until Joyal picked it up (in 2002) and eventually Lurie (Higher Topos Theory in 2009).

A few more (non-precise) words about the philosophy of infinity categories.
\begin{itemize}
    \item An $\infty$-category has \tit{spaces} of morphisms (not sets), but we really only care about their weak homotopy types.
    
    \item Composition is no longer strictly defined, only \enquote{up to contractible choice}.
    
    \item \enquote{Equality} is meaningless, the role is taken up by \enquote{weak equivalences}.
    
    \item Initial objects $I$ are now characterized by the property that for all objects $T$, $\map_\Cc(I,T)$ is weakly contractible.
\end{itemize}

\section{Quasi-categories}

For $0\le k\le n$, the $k$\tbf{-horn}\alvaropls\ $\Lambda^n_k$ is the simplicial subset of $\Delta^n$ generated by \[d_0,\dots,d_{k-1},d_{k+1},\dots,d_n.\]

For $0<k<n$, $\Lambda^n_k$ is called an \tbf{inner horn}; $\Lambda^n_0$ and $\Lambda^n_n$ are the \tbf{outer horns}.

A \tbf{horn} in a simplicial set $X$ is a morphism $\alpha:\Lambda^n_k\to X$; a horn has a \tbf{filler} if there is a morphism $\beta:\Delta^n\to X$ such that $\beta|_{\Lambda^n_k}=\alpha$.

\[some\ diagram\ here\]

A simplicial set is a \tbf{Kan complex} if all its horns have fillers. A simplicial set is a \tbf{quasi-category} if all its inner horns have fillers.

There is a more concrete combinatorial reinterpretation of the Kan condition: given a morphism $\alpha:\Lambda^n_k\to X$, this is equivalent to the data
\[x_0=\alpha(d_0),\cdots,x_{k-1}=\alpha(d_{k-1}),x_{k+1}=\alpha(d_{k+1}),\cdots,x_n=\alpha(d_n)\]
which satisfy $d_i^*(x_j)=d_{j-1}^*(x_i)$ for all $0\le i<j\le n-1$ and $i,j\ne k$. Then a horn filler $\beta:\Delta^n\to X$ is equivalent to $y=\beta(\id_{[n]})$ which satisfies $d_i^*(y)=x_i$ for all $0\le i\le n$ and $i\ne k$.

\begin{example}
Let $C$ be a category. The nerve $NC$ of $C$ is a quasi category. A morphism $\alpha:\Lambda^2_1\to NC$ is just a pair of composable morphism
\[(x\xto{f}y\xto{g}z)=\beta\in (NC)_2,\]
then clearly the diagram
\[
\begin{tikzcd}[column sep=small]
& y \ar[dr,"g"] & \\
x \ar[ur,"f"] \ar[rr,"g\circ f"] & & z
\end{tikzcd}
\]
shows that there is a filler.

For $n\ge3$, any horn contains the $1$-skeleton of $\Delta^n$. This means that fillers in $NC$ are unique, if they exist. The upshot is that in $NC$ all inner horns have unique fillers.

Then we can see that $NC$ is a Kan complex if and only if $C$ is a groupoid.

Note that later we will see that Kan complexes are the $\infty$-groupoids.
\end{example}

Quasi-categories \enquote{generalize}, i.e. contain as a full subcategory, the usual categories.

\begin{proposition}
The functor $N:\Cat\to qCat\subset\sSet$ is fully faithful with image those simplicial sets that have unique fillers for inner horns.
\end{proposition}

The proof is not too difficult, according to the Professor, we should be able to reconstruct it ourselves.

Some more terminology. If $\Cc$ is a quasi-category, the vertices $\Cc_0$ are the objects of $\Cc$, the edges $\Cc_1$ are the morphisms of $\Cc$:
\[d_1^*(f)=x\xto{f}y=d_0^*(f),\  s_0^*(x)=\id_x.\]
The $2$-simplices $\sigma\in\Cc_2$ are witnesses/homotopies for $d_1^*(\sigma)$ a composition of $d_2^*(\sigma)$ and $d_0^*(\sigma)$
\[
\begin{tikzcd}[column sep=small]
& 1 \ar[dr,"g"] & \\
0 \ar[ur,"f"] \ar[rr,"h"] & & 2
\end{tikzcd}
\]
i.e. $\de\sigma=(d_0^*(0),d_1^*(\sigma),d_2^*(\sigma))$.

Let $K,L$ be simplicial sets. The mapping simplicial set $\map(K,L)$ is defined by
\[\map(K,L)_n=\Hom_\sSet(K\times\Delta^n,L),\]
where $\alpha^*:\map(K,L)_n\to\map(K,L)_m$ is
\[\alpha^*(f:K\times\Delta^n\to L):=f\circ(K\times\alpha_*).\]
There is a natural bijection of simplicial sets
\[\map(A,\map(K,L))\cong\map(A\times K,L).\]
If $L$ is a Kan complex, then the map
\begin{align*}
    |\map(K,L)|&\to\map(|K|,|L|)\\
    [f:K\times\Delta^n\to L,t\in\sx{n}]&\mapsto\cb{|K|\xto{(-,t)}|K|\times\sx{n}\cong|K\times\Delta^n|\xto{|f|}|L|}
\end{align*}
is a weak homotopy equivalence.

The following theorem is not easy to prove.

\begin{theorem}[Joyal]
A simplicial set $X$ is a quasi-category if and only if the restriction morphism
\[\map(\Delta^2,X)\to\map(\Lambda^2_1,X)\]
is a Kan fibration and a weak equivalence.
\end{theorem}

Kan fibrations are the simplicial analogue of Serre fibrations, i.e. they have the relative lifting property for all horns
\[
\begin{tikzcd}
\Lambda^n_k \ar[d] \ar[r,"\alpha"] & X \ar[d]\\
\Delta^n \ar[r,"\Lambda"] & B
\end{tikzcd}
\]

\subsection{The Homotopy Category of a Quasi-Category}

Let $\Cc$ be a quasi-category.

Let $f,g:x\to y$ be $\Cc$-morphisms. A homotopy from $f$ to $g$ is a $2$-simplex $\sigma\in\Cc_2$ such that
\[\de\sigma=(d_0^*(\sigma),d_1^*(\sigma),d_2^*(\sigma))=(g,f,\id_x=s_0^*(x)=s_0^*(d_1^*(f))=s_0^*(d_1^*(g)))\]\rightnote{We will see that no choice is involved in this definition...}

\[
some\ diagrams\ here
\]

\begin{proposition}
Being homotopic is an equivalence relation on the set of $\Cc$-morphisms from $x$ to $y$. Moreover, $f$ is homotopic to $g$ if and only if there is a $\tau\in\Cc_2$ such that $\de\tau=(\id_y,g,f)$.

\[
some\ diagrams\ here
\]

\end{proposition}

\begin{proof}
Reflexivity. This one is easy.

Symmetry. Let $\sigma$ be a homotopy from $f$ to $g$, i.e. $\de\sigma=(g,f,\id_x)$. Then
\[(\sigma,s_0^*(g),-,s_0^*(\id_x)):\Lambda^3_2\to\Cc\]
is an inner horn.
\[some\ diagram\ here\]
Let $\mu:\Delta^3\to\Cc$ be a filler. Then $d_2^*(\mu)$ is the desired homotopy:
\[\de(d_2^*(\mu))=(f,g,\id_x).\]

Transitivity. Let $\sigma$ be a homotopy from $f$ to $g$, $\tau$ a homotopy from $g$ to $h$. Then
\[(\tau,\sigma,-,s_0^*(\id_x)):\Lambda^3_2\to\Cc\]
is an inner horn.
\[some\ diagram\ here\]
Let $\mu:Delta^3\to\Cc$ be any filler. Then $\de(d_2^*(\mu))=(h,f,\id_x)$, so $d_2^*(\mu)$ is a homotopy from $f$ to $h$.

Equivalence of the two versions of homotopy. Let $\sigma$ be a homotopy from $f$ to $g$...
\end{proof}

\begin{proposition}
Let $f:x\to y$ and $g:y\to z$ be composable morphisms in $\Cc$. Choose $\sigma\in\Cc_2$ with $\de\sigma=(g,h,f)$. Then the homotopy class of $h$ is independent of the choice of $\sigma$ and only depends on the homotopy classes of $f$ and $g$.
\end{proposition}

\begin{proof}[Partial]
Let $\tau\in\Cc_2$ be another choice of $2$-simplex with $\de\tau=(g,k,f)$. We need to show that $h$ and $k$ are homotopic. Fill the following $\Lambda^3_2$-horn.
\[some\ diagram\ here\]
\end{proof}

% Lecture 26

\lecture[Cooler stuff (mostly without proof).]{2022-01-26}

Recall: a simplicial set is a Kan complex if all its horns have fillers, a quasi-category if all its inner horns have fillers.

As an example, last time we saw that the nerve of every (usual) category has unique fillers of all inner horns.

Let $f,g:x\to Y$ be morphisms in a quasi-category. Then $f$ and $g$ are homotopic if the following four conditions hold:

\[some\ diagram\ here\]

Homotopy is an equivalence relation on the set of morphisms from $x$ to $y$.

Composition is well-defined on homotopy classes...

\begin{thmdef}
Let $\Cc$ be a quasi-category. The \tbf{homotopy category} $h\Cc$ has objects $\Cc_0$, morphisms the homotopy classes of $\Cc$-morphisms and composition as above.
\end{thmdef}

\begin{proof}
The identity property is clear.

We want to show that composition is associative: let
\[x\xto{f}y\xto{g}z\xto{i}w\]
be composable $\Cc$-morphisms. Let $\sigma,\tau\in\Cc_2$ be $2$-simplices with boundaries $\de\sigma=(g,h,f)$ (read \enquote{$h\simeq g\circ f$}) and $\de\tau(i,h,g)$. Then choose $\alpha\in\Cc_2$ with $\de\alpha=(i,j,h)$. This data provides a $\Lambda^3_2$ horn:
\[some\ diagram\ here\]
We choose a filler $\mu\in\Cc_3$ of this horn and consider $d_2^*(\mu)\in\Cc_2$. Then
\[[i]\circ([g]\circ[f])=[i]\circ[h]=[j]=[k]\circ[f]=([i]\circ[g])\circ[f]\]
where the penultimate equality is given by $d_2^*(\mu)$.
\end{proof}

\begin{example}
Let $\Cc=NC$ be the nerve of some category $C$. Then $f,g:x\to y$ are homotopic if and only if...

Hence in $NC$ \enquote{homotopy} specializes to \enquote{equality}. So $h(NC)=C$.
\end{example}

\begin{theorem}[Joyal]
A quasi-category $\Cc$ is a Kan-complex if and only if $h\Cc$ is a groupoid.
\end{theorem}

\begin{example}
There are essentially two ways of obtaining quasi-categories, from old ones (many constructions, like completion or cocompletition, slice categories et cetera, specialize to quasi-categories) or from modified nerve constructions. We focus now on the second way.

Let $\Cc$ be a $2$-category. The \tbf{Duskin nerve} $NC$ is given by:
\begin{itemize}
    \item $(NC)_0$ are the objects of $C$,
    \item $(NC)_1$ are the $1$-simplices of $C$,
    \item $(NC)_2$ are quadruples:
    \[some\ diagram\ here\]
    \item $(NC)_3$ are more complicated:
    \[some\ imprecise\ complicated\ diagram\]
    such that the following two $2$-cells are equal...
\end{itemize}
\end{example}

\begin{theorem}
The nerve of a $(2,1)$-category ($2$-category with invertible $2$-cells) is a quasi-category.
\end{theorem}

\begin{remark}
A category $C$ can be made into a $(2,1)$-category with only identity $2$-cells. In this case the Duskin nerve specializes to the ordinary nerve
\end{remark}

\begin{example}
Categories can be considered as $1$-categories (ignoring natural transformations) and they form a category $\Cat_1$, or as $2$-categories and they form a category $\Cat_2$.
We have
\[hN(\Cat_1)=\Cat_1\]
hence two categories are isomorphic if and only if they are isomorphic in the usual sense.
But in
\[h(N^\text{Duskin}(\Cat_{(2,1)}))\]
two categories are isomorphic if and only if they are equivalent in the usual sense.
\end{example}

Let $C$ be a category enriched in Kan complexes, i.e. a set of objects $\ob(C)$, Kan complexes of morphisms $\map_C(x,y)$ for all $x,y\in\ob(C)$, composition of morphisms
\[\circ:\map_C(y,z)\times\map_C(x,y)\to\map_C(x,z),\]
identities $1_x\in\map_C(x,x)_0$, that is (strictly) associative and unital.

\begin{example}
The category $\Kan$: objects are all Kan complexes,
\[\map_\Kan(X,Y)=\map(X,Y)\]
where $\map(X,Y)$ is the mapping simplicial set, with simplices
\[\map(X,Y)_n=\Hom_\sSet(X\times\Delta^n,Y).\]
If $Y$ is Kan, so is $\map(X,Y)$.
\end{example}

The \tbf{coherent nerve} of a category $C$ enriched in Kan complexes is:
\begin{itemize}
    \item $(NC)_0$ are the objects of $C$,
    \item $(NC)_1$ is the set of $1$-simplices in all mapping complexes,
    \[\Hom_{NC}(x,y)=\map_C(x,y)_1,\]
    \item $(NC)_2$ is the set of triples $(f,g,h,\tau)$...
\end{itemize}

\begin{proposition}
Let $K$ and $\Cc$ be simplicial sets.
\begin{numerate}
\item If $\Cc$ is a Kan complex, then so is $\map(K,\Cc)$.
\item If $\Cc$ is a quasi-category, then so is $\map(K,\Cc)$, the quasi-category of functors $K\to\Cc$.
\end{numerate}
\end{proposition}

The $\infty$-category of spaces is the quasi-category $N^\text{coh}(\Kan)$ and $hN^\text{coh}(\Kan)=\Ho(\sSet_\text{Kan})$.

The \tbf{core} of a quasi-category $\Cc$ is the simplicial subset of $\Cc$ with
\[(\core\Cc)_n=\cb{\sigma\in\Cc_n:\text{all edges of }\sigma\text{ are equivalent}}.\]

\begin{theorem}
$\core\Cc$ is a quasi-category, and in fact a Kan complex.
\end{theorem}

The $\infty$-category of $\infty$-categories is the coherent nerve of the Kan enriched category with objects all quasi-categories and morphisms simplicial sets $\core\map(\Cc,\Dd)$. Then $\ob(\Cat_\infty)$ are all quasi-categories, $\map_{\Cat_\infty}(\Cc,\Dd)$ are functors $\Cc\to\Dd$...

A functor $f:\Cc\to\Dd$ of quasi-categories is an equivalence if for every simplicial set $K$, the functor
\[f_*:h\map(K,\Cc)\to h\map(K,\Dd)\]
is an equivalence of ($1$-)categories.

Note: if $K=*$, $\map(*,\Cc)\cong\Cc$, then $f_*:h\Cc\to h\Dd$ is an equivalence. But equivalence of homotopy categories is not sufficient: the projection
\begin{align*}
    \Cc&\to N(h\Cc)\\
    x&\mapsto x\\
    x\xto{f}y&\mapsto[f]
\end{align*}
induces an equivalence of homotopy categories. But this is typically \tit{not} an equivalence of categories.

A functor $f:\Cc\to\Dd$ of quasi-categories is a \tbf{localization} of a set $\Ww$ of $\Cc$-morphisms if for all quasi-categories $\Dd$, the resulting functor
\[\gamma:\Fun(\Cc_\Ww,\Dd)\to\Fun(\Cc,\Dd)\]
is an equivalence (of quasi-categories!) onto the full subcategory of functors that send $\Ww$ to equivalences. 

Localization of quasi-categories exist and are unique up to equivalences under $\Cc$.

\begin{theorem}
The following quasi-categories are equivalent:
\begin{itemize}[label={-}]
    \item the $\infty$-category of spaces $N^\text{coh}(\Kan)$,
    \item the localization of $N(\sSet)$ at the weak equivalences,
    \item the localization of $N(\Top)$ at the weak homotopy equivalences,
    \item $N^\text{coh}(\Top_\text{CW})$ Kan-enriched by $\map(X,Y)=\S(\map(X,Y))$.
\end{itemize}
\end{theorem}


%%% Appendix

\chapter{Appendix}

\section{Exercises}

In this section we collect the weekly exercises we were given during the semester. Originally I wanted to provide sketches of the important or more interesting ones, but it's not gonna happen anymore for lack of time.

\subsection{AT1Sheet1}

The first exercise is about an inherently interesting result.

\unnumpar{AT1Sheet1.1}\label{exercise:AT1Sheet1.1}
Let $X$ be a finite CW-complex and $Y$ a space such that for every basepoint $y\in Y$ and every number $i$ that is less than or equal to the dimension of $X$ the set $\pi_i(Y,y)$ is finite. Show that the set $[X,Y]$ of homotopy classes of maps from $X$ to $Y$ is finite.

\begin{sketch}
\todo[inline,color=yellow]{To be added}
\end{sketch}

In the first lecture \hyperref[paragraph:hurewicz-morphism]{we introduced the pinch map}, which can be used to describe the group structure on the higher homotopy groups. In the second exercise, we describe such a map, we prove its key property (needed to show that the Hurewicz map is a morphism of groups) and we show that it is unique up to homotopy.

\unnumpar{AT1Sheet1.2}\label{exercise:AT1Sheet1.2}
For $n \ge 1$ consider the maps
\[q_1, q_2 : S^n\vee S^n\to S^n\]
where $q_i$ is the identity on the $i$-th wedge summand and sends the other wedge summand to the basepoint. A \tbf{pinch map} is a continuous map $p : S^n\to S^n\vee S^n$ such that both
composites $q_1 \circ p$ and $q_2 \circ p$ are homotopic to the identity map of $S^n$.
\begin{enumerate}
    \item[(a)] Show that there is a pinch map for every $n\ge 1$.
    \item[(b)] Show that the effect of a pinch map on singular homology is given by
    \[ p_*(x) = (i_1)_*(x) + (i_2)_*(x)\]
    for all coefficient groups $A$ and all $x\in H_n(S^n,A)$, where $i_1, i_2 : S^n \to S^n\vee S^n$ are the two wedge summand inclusions.
    \item[(c)] Suppose now that $n\ge 2$. Use the Hurewicz theorem to show that any two pinch maps are homotopic as based maps.
\end{enumerate}

\begin{sketch}
\todo[inline,color=yellow]{To be added}
\end{sketch}

The next exercise serves as a reality check: the second relative higher homotopy group is not, in general, abelian, but the quotient of it we consider in the proof of the relative Hurewicz theorem is.

\unnumpar{AT1Sheet1.3}\label{exercise:AT1Sheet1.3}
Let $A$ be a subspace of a space $X$ and let $\pi_2(X,A,x_0)^\dagger$ be the factor
group of $\pi_2(X,A,x_0)$ by the normal subgroup generated by all elements of the form \[(\omega * f) \cdot f^{-1}\]
for all $\omega\in\pi_1(A,x_0)$ and all $f\in\pi_2(X,A,x_0)$. Show that the factor group $\pi_2(X,A,x_0)^\dagger$ is abelian.

\begin{sketch}
\todo[inline,color=yellow]{To be added}
\end{sketch}

\subsection{AT1Sheet2}

An example of a non-trivial fundamental group action.

\unnumpar{AT1Sheet2.1}\label{exercise:AT1Sheet2.1}
This exercise shows, among other things, that the higher homotopy groups of a finite CW-complex need not be finitely generated (in contrast to the integral homology groups).
Let $X = S^{1} \vee S^{2}$ be the onepoint union of a circle and a $2$-sphere.
\begin{enumerate}
    \item[(a)] Find a universal cover of $X$ and calculate $\pi_{1}(X, x_{0})$ as the deck transformation group of the universal cover.
    \item[(b)] Use the Hurewicz theorem for the universal cover to calculate the group $\pi_{2}(X, x_{0})$.
    \item[(c)] Define a family of continuous maps $S^{2} \rightarrow X$ whose homotopy classes form a basis of $\pi_{2}(X, x_{0})$. Determine the action of the fundamental group $\pi_{1}(X, x_{0})$ on $\pi_{2}(X, x_{0})$ in terms of your basis.
\end{enumerate}

\begin{sketch}
\todo[inline,color=yellow]{To be added}
\end{sketch}

An application of Hurewicz theorem.

\unnumpar{AT1Sheet2.2}\label{exercise:AT1Sheet2.2}
Let $X$ be an acyclic CW-complex, i.e., all reduced integral homology groups of $X$ are trivial. Show that the suspension of $X$ is contractible.

\begin{sketch}
\todo[inline,color=yellow]{To be added}
\end{sketch}

In the following exercise we study the mapping telescope, a construction which is related to the notion of homotopy colimit.

\unnumpar{AT1Sheet2.3}\label{exercise:AT1Sheet2.3}
Let 
\begin{center}
\begin{tikzcd}
X_{0} \arrow[r, "f_{0}"] & X_{1} \arrow[r, "f_{1}"] & \cdots \arrow[r, "f_{n - 1}"] & X_{n} \arrow[r, "f_{n}"] & \cdots
\end{tikzcd}
\end{center}
be a sequence of topological spaces and continuous maps. The \textit{mapping telescope} of the sequence is the space
\begin{equation*}
    \tel_{n} X_{n} = \Bigl\{ \bigsqcup_{n \geq 0} X_{n} \times [0, 1] \Bigr\} / \sim,
\end{equation*}
where $\sim$ denotes the equivalence relation generated by
\begin{equation*}
    (x, 1) \sim (f_{n}(x), 0)
\end{equation*}
for all $n \geq 0$ and all $x \in X_{n}$. We define continuous maps
\begin{equation*}
    i_{k} : X_{k} \rightarrow \tel_{n} X_{n}
\end{equation*}
be letting $i_{k}(x)$ be the equivalence class of $(x, 0)$.
\begin{enumerate}
    \item[(a)] Show that the maps
    \begin{equation*}
        (i_{k})_{*} : H_{*}(X_{k}, A) \rightarrow H_{*}(\tel_{n}X_{n}, A)
    \end{equation*}
    induce an isomorphism
    \begin{equation*}
        \colim_{k}H_{*}(X_{k}, A) \rightarrow H_{*}(\tel_{n} X_{n}, A)
    \end{equation*}
    for every abelian coefficient group $A$.
    \item[(b)] Let $x_{0} \in X_{0}$ be a basepoint and define $x_{n} \in X_{n}$ inductively by $x_{n} = f_{n - 1}(x_{n-1})$. Use the maps
    \begin{equation*}
        (i_{k})_{*} : \pi_{m}(X_{k}, x_{k}) \rightarrow \pi_{m}(\tel_{n}X_{n}, (i_{k})(x_{k}))
    \end{equation*}
    to define an isomorphism
    \begin{equation*}
        \colim_{k} \pi_{m}(X_{k}, x_{k}) \rightarrow \pi_{m}(\tel_{n}X_{n}, (i_{0})(x_{0}))
    \end{equation*}
    Some care has to be taken here with basepoints ...
\end{enumerate}

\begin{sketch}
\todo[inline,color=yellow]{To be added}
\end{sketch}

\subsection{AT1Sheet3}

There is a fundamental adjunction between the geometric realization functor and the singular simplicial set functor that we will use throughout the course (especially in chapters \ref{chapter:hurewicz} and \ref{chapter:the-cool-chapter}).

\unnumpar{AT1Sheet3.1}\label{exercise:AT1Sheet3.1}
Let $X$ be a simplicial set, $T$ a topological space and $f : X\to \mathcal{S}(T)$ a
morphism of simplicial sets, where $\mathcal{S}(-)$ denotes the singular simplicial set.
\begin{itemize}
    \item[(a)] Consider the continuous map
    \begin{align*}
        \bigsqcup_{n\ge0} X_n\times\nabla^n &\to T\\
        X_n\times\nabla^n\ni (x,t) &\mapsto f_n(x)(t).
    \end{align*}
.    Show that this induces a continuous map $\widehat{f} : |X| \to T$ defined on the geometric realization.
    
    \item[(b)] Show that for every simplicial set $X$ and every topological space $T$ the assignment
    \begin{align*}
    \Hom_{\text{simpl. sets}}(X,\mathcal{S}(T)) &\to \Hom_{\text{top. spaces}}(|X|,T)\\
    f &\mapsto \widehat{f}
    \end{align*}
is bijective.
    
    \item[(c)] Show that the bijection of part (b) is natural in both variables.
\end{itemize}

\begin{sketch}
\todo[inline,color=yellow]{To be added}
\end{sketch}

In the second exercise we prove a very useful homeomorphism which is used in the proof of theorem \ref{theorem:simplicial-deformation-retraction} and will be generalized in \hyperref[exercise:AT1Sheet11.2]{AT1Sheet11.2}.

\unnumpar{AT1Sheet3.2}\label{exercise:AT1Sheet3.2}
Show that for every $n\ge0$ the map
\[
(|p_1|,|p_2|) : |\Delta[n] \times \Delta[1]| \to |\Delta[n]|\times|\Delta[1]|
\]
is a homeomorphism, where $p_1 : \Delta[n] \times\Delta[1]\to\Delta[n]$ and $p_2 : \Delta[n] \times\Delta[1] \to\Delta[1]$ are the
projections to the two factors.

\begin{sketch}
\todo[inline,color=yellow]{To be added}
\end{sketch}

The nerve construction is a way of constructing simplicial sets from small categories (in \hyperref[exercise:AT1Sheet4.1]{AT1Sheet4.1} we will use it to construct a simplicial set out of a group).

\unnumpar{AT1Sheet3.3}\label{exercise:AT1Sheet3.3}
Let $I$ be a small category. The nerve of $I$ is the simplicial set $NI$ given by
\[
(NI)_n = \text{ set of all composable $n$-tuples of morphisms in } I.
\]
\begin{enumerate}
    \item[(a)] Show that there is a unique way to extend this data to a simplicial set in such a way that the face and degeneracy morphisms are given by the following formulas. For $n\ge1$ and $0\ge i\ge n$, the $i$-th boundary map $d_i : (NI)_n \to (NI)_{n-1}$ is defined by
    \[d_i(f_n,...,f_1) = \begin{cases}
        (f_n,...,f_2) & i = 0,\\
        (f_n,...,f_{i+2},f_{i+1}\circ f_i,f_{i-1},...,f_1) &0 < i < n,\\
        (f_{n-1},...,f_1) &i = n.\end{cases}\]
    For $n\ge0$ and $0\le i\le n$, the degeneracy map $s_i : (NI)_n\to (NI)_{n+1}$ is given by
    \[s_i(f_n,...,f_1) = (f_n,...,f_{i+1},\id,f_i,...,f_1).\]
    \item[(b)] Let $J$ be another small category and $F : I\to J$ a functor. We define maps $(NF)_n : (NI)_n\to (NJ)_n$ by
    \[(NF)(f_n,...,f_1) = (F(f_n),...,F(f_1)).\]
    Shows that these maps form a morphism $NF : NI\to NJ$ of simplicial sets.
    
    \item[(c)] Show that the nerve construction preserves products, i.e., that the morphism
    \[
    N(I \times J) \xrightarrow{(N_{\operatorname{proj}_I} , N_{\operatorname{proj}_J} )}NI \times NJ
    \]
    is an isomorphism of simplicial sets.
    
    \item[(d)] Let $[1]$ denote the category with two objects 0 and 1 and three morphisms $\id_0, \id_1$ and $f : 0\to 1$. Let $\tau : F\to G$ be a natural transformation of functors from $I$ to $J$. Define a functor $H : I\times[1]\to J$ satisfying $H(-,0) = F$, $H(-,1) = G$ and $H(i,f) = \tau_i : F(i)\to G(i)$. Show that the natural transformation $\tau$ yields a simplicial homotopy between the morphisms $NF, NG : NI\to NJ$.

    \item[(e)] Show that the nerves $NI$ and $NJ$ of two equivalent small categories are homotopy equivalent simplicial sets.
\end{enumerate}

\begin{sketch}
\todo[inline,color=yellow]{To be added}
\end{sketch}

\subsection{AT1Sheet4}

The next exercise shows a way to construct the classifying space of a group $G$, a path-connected CW-complex $K(G,1)$ with fundamental group isomorphic to $G$ and all higher homotopy groups trivial (even though we do not prove all the relevant facts about it in the exercise). This is known as the bar construction.

\unnumpar{AT1Sheet4.1}\label{exercise:AT1Sheet4.1}
Let $G$ be a group. For $n\ge0$, let $(BG)_n = G^n$ be the cartesian product of $n$ copies of the underlying set of $G$. For $n\ge1$ and $0\le i\le n$ define $d_i : (BG)_n \to (BG)_{n-1}$ by
$(g_n,...,g_1) \mapsto (g_n,...,g_i,g_{i+1}\cdot g_i,g_{i-1},...,g_1)$. For $n\ge1$ and $0\le i\le n -1$ define
$s_i : (BG)_{n-1}\mapsto(BG)_n$ by
$s_i(g_{n-1},...,g_1) = (g_{n-1},...,g_{i+1},1,g_i,...,g_1)$.
\begin{enumerate}
    \item[(a)] Show that $BG$ extends to a simplicial set. Identify $BG$ as the nerve, in the sense of \hyperref[exercise:AT1Sheet3.3]{AT1Sheet3.3}, of a suitable category.
    \item[(b)] In the geometric realization $|BG|$ we take the class of $(e,1)\in(BG)_0 \times\nabla^0$ as the basepoint and call it `e'. Every element $g\in G = (BG)_1$ yields a continuous map
    \[\{g\}\times\nabla^1 \xhookrightarrow{\text{inclusion}} \bigcup_{n\ge0} (BG)_n \times\nabla^n \xrightarrow{\text{projection}} |BG|.\]
    Show that this map takes the two boundary points of $\{g\}\times\nabla^1$ to the basepoint of $|BG|$.
    \item[(c)] We identify the interval $[0,1]$ with $\{g\}\times\nabla^1$ via the homeomorphism sending t to $(g,(t,1 -t))$. By part (b) the composition
    \[[0,1]\xrightarrow{\cong}\{g\}\times\nabla^1 \to |BG|\]
    is a loop at the basepoint $e\in |BG|$. We let $w(g)$ denote the homotopy class of this loop in the fundamental group $\pi_1(|BG|,e)$. Show that
    \[w:G\to\pi_1(|BG|,e)\]
    is a group homomorphism.
\end{enumerate}

\begin{sketch}
\todo[inline,color=yellow]{To be added}
\end{sketch}

A cheap Poincaré conjecture.

\unnumpar{AT1Sheet4.2}\label{exercise:AT1Sheet4.2}
Show that every compact simply-connected 3-manifold without boundary
is homotopy equivalent to $S^3$. (Hint: you might want to use Poincaré duality and the
Hurewicz theorem.)

\begin{sketch}
\todo[inline,color=yellow]{To be added}
\end{sketch}

Mapping tori.

\unnumpar{AT1Sheet4.3}\label{exercise:AT1Sheet4.3}
To be added.

\begin{sketch}
\todo[inline,color=yellow]{To be added}
\end{sketch}

\subsection{AT1Sheet5}

Using the long exact sequence associated to a Serre fibration (theorem \ref{theorem:long-exact-sequence-serre-fibration}) we can compute some interesting homotopy groups.

\unnumpar{AT1Sheet5.1}\label{exercise:AT1Sheet5.1}
We denote by $V_{n}(\R^{k})$ the Stiefel manifold of $n$-frames in $\R^{k}$. An element of $V_{n}(\R^{k})$ is an $n$-tuple $(x_{1}, \ldots, x_{n})$ of orthonormal vectors from $\R^{k}$. The set $V_{n}(\R^{k})$ is topologized as a subspace of $(\R^{k})^{n}$. We denote by $Gr_n(\R^k)$ the Grassmann manifold of $n$-dimensional vector subspaces of $\R^{k}$. The set $Gr_n(\R^k)$ carries the quotient topology with respect to the map $q: V_n(\R^k) \rightarrow Gr_n(\R^k)$ that takes an $n$-frame onto its span. The complex Stiefel and Grassmann manifolds $V_{n}(\CC^{k})$ and $Gr_{n}(\CC^{k})$ are defined analogously. Show that the following maps are fiber bundles and identify the fibers:
\begin{equation*}
\begin{cases}
q: V_n(\R^k) \rightarrow Gr_n(\R^k) \quad \text{for } 1 \leq n \leq k, \\
q: V_{n}(\CC^{k}) \rightarrow Gr_{n}(\CC^{k}) \quad \text{for } 1 \leq n \leq k, \\
p: V_n(\R^k) \rightarrow V_{m}(\R^{k}) \quad \text{for } 1 \leq m < n \leq k, \\
p: V_{n}(\CC^{k}) \rightarrow V_{m}(\CC^{k}) \quad \text{for } 1 \leq m < n \leq k,
\end{cases}
\end{equation*}
here the maps $p$ forget the last $m-n$ vectors: $p(x_{1}, \ldots, x_{n}) = (x_{1}, \ldots, x_{m})$. Use the long exact homotopy group sequences to show that $V_n(\R^k)$ is $(k - n - 1)$-connected and $V_{n}(\CC^{k})$ is $(2k - 2n)$-connected. Calculate $\pi_{k-n}(V_n(\R^k))$ and $\pi_{2k - 2n + 1}(V_{n}(\CC^{k}))$.

\begin{sketch}
\todo[inline,color=yellow]{To be added}
\end{sketch}

\unnumpar{AT1Sheet5.2}\label{exercise:AT1Sheet5.2}
Let $G$ be a topological group that is also a Hausdorff space. Let $H$ be a subgroup of $G$ and $G/H$ the space of right cosets with the quotient topology. Show:
\begin{enumerate}
    \item[(a)] If $H$ is closed in $G$, then $G/H$ is a Hausdorff space.
    \item[(b)] Let $H$ be closed in $G$ and suppose that there is a \textit{local section} i.e., a neighbourhood $U$ if $1\cdot H$ in $G/H$ and a continuous section $\sigma: U \rightarrow G$ (i.e., $p \circ \sigma = \id_{U}$). Let $K$ be a closed subgroup of $H$. Then the projection
    \begin{equation*}
        G/K \rightarrow G/H, \quad gK \mapsto gH
    \end{equation*}
    is a fibre bundle with fibre $H/K$.
\end{enumerate}

\begin{sketch}
\todo[inline,color=yellow]{To be added}
\end{sketch}

\unnumpar{AT1Sheet5.3}\label{exercise:AT1Sheet5.3}
Let $p: E \rightarrow B$ be a fiber bundle with path connected base, $F = p^{-1}(b)$ the fiber over a point $b \in B$, and $x \in F$. Suppose that the inclusion $F \rightarrow E$ is homotopic to a constant map.

Show that the long exact homotopy group sequence degenerates into an isomorphism between $\pi_{n}(B, b)$ and $\pi_{n}(E, x) \times \pi_{n - 1}(F, x)$ for all $n \geq 1$. Apply this to the Hopf fibrations $\nu: S^{7} \rightarrow S^{4}$ and $\sigma: S^{15} \rightarrow S^{8}$ to deduce that the groups $\pi_{7}(S^{4}, z)$ and $\pi_{15}(S^{8}, z)$ each contain a copy of $\Z$ as a direct summand.

\begin{sketch}
\todo[inline,color=yellow]{To be added}
\end{sketch}

\subsection{AT1Sheet6}

\unnumpar{AT1Sheet6.1}\label{exercise:AT1Sheet6.1}
Let $p : S^m \to S^n$ be a fiber bundle with $m,n\ge1$ whose fiber is homeomorphic to the sphere $S^k$. Show that then $k=n-1$ and $m=2n-1$.

\begin{sketch}
\todo[inline,color=yellow]{To be added}
\end{sketch}

\unnumpar{AT1Sheet6.2}\label{exercise:AT1Sheet6.2}
Show by examples that various hypotheses in the exponential law are really necessary.
\begin{enumerate}
    \item[(a)] Find spaces $X$ and $Z$ such that the evaluation map
    \[
    \ev : Z^X\times X \to Z,\quad (f,x) \mapsto f(x)
    \]
    is not continuous.
    \item[(b)] Find spaces $X$, $Y$ and $Z$ such that the exponential map
    \[\Phi : Z^{X\times Y} \to (Z^X)^Y\]
    is not surjective.
    \item[(c)] Find spaces $X$, $Y$ and $Z$ such that $X$ is locally compact but the exponential map $\Phi$ is not a homeomorphism.
\end{enumerate}

\begin{sketch}
\todo[inline,color=yellow]{To be added}
\end{sketch}

\unnumpar{AT1Sheet6.3}\label{exercise:AT1Sheet6.3}
Let $X$ be a topological space with basepoint $x_0$. Let
\[
    E = \{f \in X^{\nabla^2}\ |\ f(1,0,0) = f(0,1,0) = f(0,0,1) = x_0\}
\]
be the space of continuous maps from the 2-simplex to $X$ that takes all three vertices to the basepoint. We define continuous maps
\(i,j : [0,1] \to\nabla^2\)
by \(i(t) = (t,1-t,0)\), respectively \(j(t) = (0,t,1-t)\).
Show that the map
\[\Phi: E \to\Omega X \times \Omega X,\quad f \mapsto (f\circ i,f\circ j)\]
is a homotopy equivalence.

\begin{sketch}
\todo[inline,color=yellow]{To be added}
\end{sketch}

\subsection{AT1Sheet7}

\unnumpar{AT1Sheet7.1}\label{exercise:AT1Sheet7.1}
Let $X$ be a topological space and $Z$ a Hausdorff space. Let $f: X \rightarrow Z$ be a continuous map and $\{f_{n}\}_{n \geq 1}$ a sequence in $Z^{X}$. Suppose that $f$ is a limit of the sequence $\{f_{n}\}$ in the compact-open topology on $Z^{X}$. Show that then the sequence $\{f_{n}\}$ converges pointwise to $f$. Show by example that the converse need not hold.

\begin{sketch}
\todo[inline,color=yellow]{To be added}
\end{sketch}

\unnumpar{AT1Sheet7.2}\label{exercise:AT1Sheet7.2}
Let $p : E\to B$ be a Serre fibration whose total space is contractible.
Show that the fiber of $p$ over a point $b\in B$ is weakly homotopy equivalent to the loop space of $B$, based at $b$.

\begin{sketch}
\todo[inline,color=yellow]{To be added}
\end{sketch}

\unnumpar{AT1Sheet7.3}\label{exercise:AT1Sheet7.3}
Given a permutation $\sigma \in \Sigma_3$ of the set $\{1,2,3\}$, define the homeomorphism
\[
    \sigma_* : \nabla^2 \to \nabla^2 \quad\text{by}\quad \sigma_*(x_1,x_2,x_3) = (x_{\sigma(1)},x_{\sigma(2)},x_{\sigma(3)}).
\]
For each of the six elements of the group $\Sigma_3$, identify the composite
{\small
\[
    \pi_1(X) \times\pi_1(X)\!\cong\!\pi_0(\Omega X \times\Omega X) \xrightarrow{\pi_0(\Phi)^{-1}}
    \pi_0(E) \xrightarrow{\pi_0(\sigma_*)}
    \pi_0(E) \xrightarrow{\pi_0(\Phi)}
    \pi_0(\Omega X \times\Omega X)\!\cong\!\pi_1(X) \times\pi_1(X)
\]}
explicitly in terms of group theoretic operations. Here $\Phi : E \to\Omega X \times\Omega X$ is the homotopy
equivalence introduced in \hyperref[exercise:AT1Sheet6.3]{AT1Sheet6.3}, and $\sigma_* : E \to E$ is the homeomorphism obtained
by restricting $(\sigma_*)^* : X^{\nabla^2}
\to X^{\nabla^2}$ to $E$.

\begin{sketch}
\todo[inline,color=yellow]{To be added}
\end{sketch}

\subsection{AT1Sheet8}

In \hyperref[exercise:AT1Sheet2.3]{AT1Sheet2.3} we studied a construction related to the homotopy colimit, now we study the homotopy limit.

\unnumpar{AT1Sheet8.1}\label{exercise:AT1Sheet8.1}
Let
\[
\cdots \to P_{n+1}
\xrightarrow{p_n} P_n
\xrightarrow{p_{n-1}} P_{n-1}
\xrightarrow{p_{n-2}} \cdots
\xrightarrow{p_0} P_0
\]
be a sequence of topological spaces and continuous maps. The homotopy limit of the
sequence is the space
\[
\holim_n P_n =
\left\{ (\omega_n)_{n\ge0} \in\prod_{n\ge0} P_n^{[0,1]}\ \Bigg|\ \omega_n(1) = p_n(\omega_{n+1}(0)) \text{ for all } n\ge0\right\}.
\]
The homotopy limit has the subspace topology of the product topology. We define continuous maps
\[
q_k : \holim_n P_n \to P_k \quad\text{ by }\quad q_k((\omega_n)_{n\ge0}) = \omega_k(0).\]
Let $\omega = (\omega_n)$ be a basepoint in $\holim_nP_n$ and let $m\ge1$.
\begin{enumerate}
    \item[(a)] We define a homomorphism $\Psi_n$ as the composite 
    \[
    \pi_m(P_{n+1},\omega_{n+1}(0))) \xrightarrow{(p_n)_*}
    \pi_m(P_n,p_n(\omega_{n+1}(0))) \xrightarrow{(\omega_n)_*} \pi_m(P_n,\omega_n(0)).
    \]
Here $(\omega_n)_*$ is the isomorphism given by conjugation with the path $\omega_n$. Show that
the homomorphisms
    \[
    \pi_m(q_k) : \pi_m(\holim_nP_n,\omega) \to \pi_m(P_k,\omega_k(0))
    \]
satisfy $\Psi_k \circ \pi_m(q_{k+1}) = \pi_m(q_k)$
and hence they assemble into a homomorphism
\begin{equation}
\pi_m(\holim_nP_n,\omega) \to \lim_n \pi_m(P_n,\omega_n(0)),
\end{equation}
where the limit in the right hand side is taken over the homomorphisms $\Psi_n$.
\item[(b)] Show the homomorphism (1) defined in (a) is surjective.
\item[(c)] Suppose that there is an $N\ge0$ such that for all $n\ge N$ the map
\[
\Psi_n : \pi_{m+1}(P_{n+1},\omega_{n+1}(0)) \to \pi_{m+1}(P_n,\omega_n(0))
\]
is surjective. Show that then the homomorphism (1) is bijective.
\end{enumerate}

\begin{sketch}
\todo[inline,color=yellow]{To be added}
\end{sketch}

\unnumpar{AT1Sheet8.2}\label{exercise:AT1Sheet8.2}
Prove the following.
\begin{enumerate}
    \item[(a)] Let
    \begin{center}
    \begin{tikzcd}
G_{0} \arrow[r, "\alpha_{0}"] & G_{1} \arrow[r, "\alpha_{1}"] & \ldots \arrow[r, "\alpha_{m-1}"] & G_{m} \arrow[r, "\alpha_{m}"] & \ldots
\end{tikzcd}
\end{center}
be a sequence of groups and groups homomorphisms. Consider $n \geq 1$; if $n \geq 2$ assume that all groups $G_{m}$ are abelian. Let $X_{m}$ be an Eilenberg-Maclane space of type $K(G_{m}, n)$ and $f_{m}: X_{m} \rightarrow X_{m + 1}$ a continuous based map that realizes $\alpha_{m}$ on $\pi_{n}$. Show that the mapping telescope of the sequence $\{f_{m}\}$ is an Eilenberg-Maclane space of type $K(G_{\infty}, n)$, where $G_{\infty}$ is a colimit of the original sequence of group homomorphisms.
    \item[(b)] Let $f: S^{1} \rightarrow S^{1}$ be the standard degree $n$ map on the circle defined by $f(z) = z^{n}$. Show that the mapping telescope of the sequence
    \begin{center}
    \begin{tikzcd}
S^{1} \arrow[r, "f"] & S^{1} \arrow[r, "f"] & \ldots \arrow[r, "f"] & S^{1} \arrow[r, "f"] & \ldots
\end{tikzcd}
    \end{center}
    is an Eilenberg-Maclane space and describe its fundamental group.
\end{enumerate}

\begin{sketch}
\todo[inline,color=yellow]{To be added}
\end{sketch}

\unnumpar{AT1Sheet8.3}\label{exercise:AT1Sheet8.3}
Let $A$ be an abelian group and $n\ge2$. Show that the homology group
$H_{n+1}(K(A,n),\Z)$
is trivial. (Hint: construct an Eilenberg-MacLane space from a \tbf{Moore space}, i.e. a space with only one prescribed non-trivial reduced homology group.)

\begin{sketch}
\todo[inline,color=yellow]{To be added}
\end{sketch}

\subsection{AT1Sheet9}

\unnumpar{AT1Sheet9.1}\label{exercise:AT1Sheet9.1}
To be added.

\begin{sketch}
\todo[inline,color=yellow]{To be added}
\end{sketch}

\unnumpar{AT1Sheet9.2}\label{exercise:AT1Sheet9.2}
To be added.

\begin{sketch}
\todo[inline,color=yellow]{To be added}
\end{sketch}

The next exercise shows a way of constructing a topological space with prescribed homotopy groups in each dimension.

\unnumpar{AT1Sheet9.3}\label{exercise:AT1Sheet9.3}
To be added.

\begin{sketch}
\todo[inline,color=yellow]{To be added}
\end{sketch}

\subsection{AT1Sheet10}

\unnumpar{AT1Sheet10.1}\label{exercise:AT1Sheet10.1}
Show that every CW-complex that is $n$-connected and $n$-dimensional is contractible.

\begin{sketch}
\todo[inline,color=yellow]{To be added}
\end{sketch}

\unnumpar{AT1Sheet10.2}\label{exercise:AT1Sheet10.2}
Show that for every continuous map $f : X \to Y$, the following conditions
are equivalent.
\begin{enumerate}
\item[(i)] The map $f$ is a weak homotopy equivalence.
\item[(ii)] For all $n \ge0$ and all continuous maps $\alpha : \partial D^n \to X$ and $\beta : D^n \to Y$ such that
$\left.\beta\right|_{\partial D^n} = f \circ\alpha$, there is a continuous map $\lambda : D^n \to X$ such that $\left.\lambda\right|_{\partial D^n} = \alpha$ and such
that $f\circ \lambda : D^n \to Y$ is homotopic, relative to $\partial D^n$, to $\beta$.
\item[(iii)] For every CW-complex $K$ and every subcomplex $L$ of $K$, all continuous maps $\alpha :
L \to X$ and $\beta : K \to Y$ such that $\left.\beta\right|_L = f\circ\alpha$, there is a continuous map $\lambda : K \to X$
such that $\left.\lambda\right|_L = \alpha$ and such that $f\circ \lambda : K \to Y$ is homotopic, relative to $L$, to $\beta$.
\item[(iv)] For every CW-complex $K$, the induced map
$[K, f]: [K, X] \to [K, Y]$
of homotopy classes of continuous maps is bijective.
\end{enumerate}

\begin{sketch}
\todo[inline,color=yellow]{To be added}
\end{sketch}

\unnumpar{AT1Sheet10.3}\label{exercise:AT1Sheet10.3}
Let $A$ and $B$ be abelian groups, and $n\ge1$.
\begin{enumerate}
    \item[(i)] Show that for $1\le m<n$ the only natural transformation $H^n(X; A) \to H^m(X; B)$ is the zero transformation.
    \item[(ii)] Every group homomorphism $\varphi : A \to B$ gives rise to a coefficient homomorphism
    \[
    \varphi_* : H^n(X; A) \to H^n(X; B)
    \]
    where $X$ is any space. Show that this assignment defines an isomorphism of groups
    \[
    \Hom(A, B) \to \Nat(H^n(-; A), H^n(-; B))
    \]
    to the group of cohomology operations of type $(A, n, B, n)$.
    \item[(iii)] In an earlier exercise we constructed from a short exact sequence of abelian groups
    \[ 0 \to B \xrightarrow{i} E \xrightarrow{p} A \to 0\]
    a Bockstein homomorphism
    \[
    \beta(i, p): H^n(X; A) \to H^{n+1}(X; B)
    \]
    where $X$ is any space. The Bockstein homomorphism only depends on the class of the extension in $\Ext(A, B)$.
    Show that this assignment defines an isomorphism of groups
    \[
    \beta : \Ext(A, B) \to \Nat(H^n(-; A), H^{n+1}(-; B))
    \]
    to the group of cohomology operations of type $(A, n, B, n + 1)$.
\end{enumerate}

\begin{sketch}
\todo[inline,color=yellow]{To be added}
\end{sketch}

\subsection{AT1Sheet11}

\unnumpar{AT1Sheet11.1}\label{exercise:AT1Sheet11.1}
Prove the following.
\begin{enumerate}
    \item[(i)]A simplicial set $X$ is $m$-dimensional if all simplices of $X_n$ for $n > m$ are degenerate.
Show that the product of an $m$-dimensional simplicial set and an $n$-dimensional
simplicial set is $(m + n)$-dimensional.
    \item[(ii)] Identify the non-degenerate simplices of the simplicial set $\Delta^n \times\Delta^1$. How many non-degenerate $(n + 1)$-simplices does $\Delta^n\times\Delta^1$ have?
    \end{enumerate}

\begin{sketch}
\todo[inline,color=yellow]{To be added}
\end{sketch}

The next exercise is an important generalization of \hyperref[exercise:AT1Sheet3.2]{AT1Sheet3.2}, which will be used often throughout chapter \ref{chapter:the-cool-chapter}.

\unnumpar{AT1Sheet11.2}\label{exercise:AT1Sheet11.2}
Prove the following.
\begin{itemize}

    \item[(i)] Let $K$ be a compact space. Show that for every space $X$ the map
    \[\eta: X\to (X\times K)^K,\ \eta(x)(k)=(x,k)\]
    is continuous.
    
    \item[(ii)] Show that for every compact space $K$, the functor $-\times K$ from $\Top$ to itself is left adjoint to the functor sending a space $Y$ to the space $Y^K$ with the compact-open topology.
    
    \item[(iii)] Show that for every simplicial set A the map
    \[(|p_1|,|p_2|):|X\times\Delta^1|\to|X|\times|\Delta^1|\]
    is a homeomorphism, where where $p_1:A\times\Delta^1\to A$ and $p_2:A\times\Delta^1\to \Delta^1$ are the projections to the two factors. (Hint: use the simplicial skeleton filtration for A and the fact that $-\times|\Delta^1|$ preserves colimits to reduce the claim to the special case A = $\Delta^n$ which was shown earlier.)
    
\end{itemize}

\begin{sketch}
\todo[inline,color=yellow]{To be added}
\end{sketch}

This is essentially the proof of proposition \ref{theorem:classifying space}.

\unnumpar{AT1Sheet11.3}\label{exercise:AT1Sheet11.3}
Let $G$ be a group, and let $X$ be a $G$-simplicial set. Suppose that the $G$-action on the set $X_0$ of vertices is free.
\begin{itemize}
    \item[(i)] Show that the action of $G$ on the set $X_n$ of $n$-simplices is free for every $n\geq0$.
    \item[(ii)] Show that the action of $G$ on the geometric realization $|X|$ is free and properly discontinuous, i.e. every point $x\in|X|$ has a neighborhood in $U$ in $X$ such that $U\cap(g\cdot U)=\emptyset$ for all $g\in G$ with $g\neq 1$.
\end{itemize}

\begin{sketch}
\todo[inline,color=yellow]{To be added}
\end{sketch}

\subsection{AT1Sheet12}

\todo[inline,color=yellow]{To be added}



\nocite{*}


\backmatter\KOMAoption{chapterprefix}{false}
\printbibliography[heading=bibintoc, title={References}]
\end{document}
