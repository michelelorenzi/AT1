% Lecture 25

\chapter{A Taste of Infinity Categories}

\lecture[Cool stuff.]{2022-01-24}

Reference: \href{https://arxiv.org/abs/1007.2925}{A Short Course on $\infty$-Categories} by Moritz Groth.

\warning\ \textcolor{red}{This is just a stub! In particular, while the material on simplicial sets in these notes might be good for beginners, I strongly advise against reading this chapter: there are many details I never got around filling (and probably never will).}

We will talk about quasi-categories, which are a model for \enquote{$\infty$-categories}/\enquote{$(\infty,1)$-categories}.

This theory began with Boardman and Vogt in 1973, but then it remained dormant until Joyal picked it up (in 2002) and eventually Lurie (Higher Topos Theory in 2009).

A few more (non-precise) words about the philosophy of infinity categories.
\begin{itemize}
    \item An $\infty$-category has \tit{spaces} of morphisms (not sets), but we really only care about their weak homotopy types.
    
    \item Composition is no longer strictly defined, only \enquote{up to contractible choice}.
    
    \item \enquote{Equality} is meaningless, the role is taken up by \enquote{weak equivalences}.
    
    \item Initial objects $I$ are now characterized by the property that for all objects $T$, $\map_\Cc(I,T)$ is weakly contractible.
\end{itemize}

\section{Quasi-categories}

For $0\le k\le n$, the $k$\tbf{-horn}\alvaropls\ $\Lambda^n_k$ is the simplicial subset of $\Delta^n$ generated by \[d_0,\dots,d_{k-1},d_{k+1},\dots,d_n.\]

For $0<k<n$, $\Lambda^n_k$ is called an \tbf{inner horn}; $\Lambda^n_0$ and $\Lambda^n_n$ are the \tbf{outer horns}.

A \tbf{horn} in a simplicial set $X$ is a morphism $\alpha:\Lambda^n_k\to X$; a horn has a \tbf{filler} if there is a morphism $\beta:\Delta^n\to X$ such that $\beta|_{\Lambda^n_k}=\alpha$.

\[some\ diagram\ here\]

A simplicial set is a \tbf{Kan complex} if all its horns have fillers. A simplicial set is a \tbf{quasi-category} if all its inner horns have fillers.

There is a more concrete combinatorial reinterpretation of the Kan condition: given a morphism $\alpha:\Lambda^n_k\to X$, this is equivalent to the data
\[x_0=\alpha(d_0),\cdots,x_{k-1}=\alpha(d_{k-1}),x_{k+1}=\alpha(d_{k+1}),\cdots,x_n=\alpha(d_n)\]
which satisfy $d_i^*(x_j)=d_{j-1}^*(x_i)$ for all $0\le i<j\le n-1$ and $i,j\ne k$. Then a horn filler $\beta:\Delta^n\to X$ is equivalent to $y=\beta(\id_{[n]})$ which satisfies $d_i^*(y)=x_i$ for all $0\le i\le n$ and $i\ne k$.

\begin{example}
Let $C$ be a category. The nerve $NC$ of $C$ is a quasi category. A morphism $\alpha:\Lambda^2_1\to NC$ is just a pair of composable morphism
\[(x\xto{f}y\xto{g}z)=\beta\in (NC)_2,\]
then clearly the diagram
\[
\begin{tikzcd}[column sep=small]
& y \ar[dr,"g"] & \\
x \ar[ur,"f"] \ar[rr,"g\circ f"] & & z
\end{tikzcd}
\]
shows that there is a filler.

For $n\ge3$, any horn contains the $1$-skeleton of $\Delta^n$. This means that fillers in $NC$ are unique, if they exist. The upshot is that in $NC$ all inner horns have unique fillers.

Then we can see that $NC$ is a Kan complex if and only if $C$ is a groupoid.

Note that later we will see that Kan complexes are the $\infty$-groupoids.
\end{example}

Quasi-categories \enquote{generalize}, i.e. contain as a full subcategory, the usual categories.

\begin{proposition}
The functor $N:\Cat\to qCat\subset\sSet$ is fully faithful with image those simplicial sets that have unique fillers for inner horns.
\end{proposition}

The proof is not too difficult, according to the Professor, we should be able to reconstruct it ourselves.

Some more terminology. If $\Cc$ is a quasi-category, the vertices $\Cc_0$ are the objects of $\Cc$, the edges $\Cc_1$ are the morphisms of $\Cc$:
\[d_1^*(f)=x\xto{f}y=d_0^*(f),\  s_0^*(x)=\id_x.\]
The $2$-simplices $\sigma\in\Cc_2$ are witnesses/homotopies for $d_1^*(\sigma)$ a composition of $d_2^*(\sigma)$ and $d_0^*(\sigma)$
\[
\begin{tikzcd}[column sep=small]
& 1 \ar[dr,"g"] & \\
0 \ar[ur,"f"] \ar[rr,"h"] & & 2
\end{tikzcd}
\]
i.e. $\de\sigma=(d_0^*(0),d_1^*(\sigma),d_2^*(\sigma))$.

Let $K,L$ be simplicial sets. The mapping simplicial set $\map(K,L)$ is defined by
\[\map(K,L)_n=\Hom_\sSet(K\times\Delta^n,L),\]
where $\alpha^*:\map(K,L)_n\to\map(K,L)_m$ is
\[\alpha^*(f:K\times\Delta^n\to L):=f\circ(K\times\alpha_*).\]
There is a natural bijection of simplicial sets
\[\map(A,\map(K,L))\cong\map(A\times K,L).\]
If $L$ is a Kan complex, then the map
\begin{align*}
    |\map(K,L)|&\to\map(|K|,|L|)\\
    [f:K\times\Delta^n\to L,t\in\sx{n}]&\mapsto\cb{|K|\xto{(-,t)}|K|\times\sx{n}\cong|K\times\Delta^n|\xto{|f|}|L|}
\end{align*}
is a weak homotopy equivalence.

The following theorem is not easy to prove.

\begin{theorem}[Joyal]
A simplicial set $X$ is a quasi-category if and only if the restriction morphism
\[\map(\Delta^2,X)\to\map(\Lambda^2_1,X)\]
is a Kan fibration and a weak equivalence.
\end{theorem}

Kan fibrations are the simplicial analogue of Serre fibrations, i.e. they have the relative lifting property for all horns
\[
\begin{tikzcd}
\Lambda^n_k \ar[d] \ar[r,"\alpha"] & X \ar[d]\\
\Delta^n \ar[r,"\Lambda"] & B
\end{tikzcd}
\]

\subsection{The Homotopy Category of a Quasi-Category}

Let $\Cc$ be a quasi-category.

Let $f,g:x\to y$ be $\Cc$-morphisms. A homotopy from $f$ to $g$ is a $2$-simplex $\sigma\in\Cc_2$ such that
\[\de\sigma=(d_0^*(\sigma),d_1^*(\sigma),d_2^*(\sigma))=(g,f,\id_x=s_0^*(x)=s_0^*(d_1^*(f))=s_0^*(d_1^*(g)))\]\rightnote{We will see that no choice is involved in this definition...}

\[
some\ diagrams\ here
\]

\begin{proposition}
Being homotopic is an equivalence relation on the set of $\Cc$-morphisms from $x$ to $y$. Moreover, $f$ is homotopic to $g$ if and only if there is a $\tau\in\Cc_2$ such that $\de\tau=(\id_y,g,f)$.

\[
some\ diagrams\ here
\]

\end{proposition}

\begin{proof}
Reflexivity. This one is easy.

Symmetry. Let $\sigma$ be a homotopy from $f$ to $g$, i.e. $\de\sigma=(g,f,\id_x)$. Then
\[(\sigma,s_0^*(g),-,s_0^*(\id_x)):\Lambda^3_2\to\Cc\]
is an inner horn.
\[some\ diagram\ here\]
Let $\mu:\Delta^3\to\Cc$ be a filler. Then $d_2^*(\mu)$ is the desired homotopy:
\[\de(d_2^*(\mu))=(f,g,\id_x).\]

Transitivity. Let $\sigma$ be a homotopy from $f$ to $g$, $\tau$ a homotopy from $g$ to $h$. Then
\[(\tau,\sigma,-,s_0^*(\id_x)):\Lambda^3_2\to\Cc\]
is an inner horn.
\[some\ diagram\ here\]
Let $\mu:Delta^3\to\Cc$ be any filler. Then $\de(d_2^*(\mu))=(h,f,\id_x)$, so $d_2^*(\mu)$ is a homotopy from $f$ to $h$.

Equivalence of the two versions of homotopy. Let $\sigma$ be a homotopy from $f$ to $g$...
\end{proof}

\begin{proposition}
Let $f:x\to y$ and $g:y\to z$ be composable morphisms in $\Cc$. Choose $\sigma\in\Cc_2$ with $\de\sigma=(g,h,f)$. Then the homotopy class of $h$ is independent of the choice of $\sigma$ and only depends on the homotopy classes of $f$ and $g$.
\end{proposition}

\begin{proof}[Partial]
Let $\tau\in\Cc_2$ be another choice of $2$-simplex with $\de\tau=(g,k,f)$. We need to show that $h$ and $k$ are homotopic. Fill the following $\Lambda^3_2$-horn.
\[some\ diagram\ here\]
\end{proof}
